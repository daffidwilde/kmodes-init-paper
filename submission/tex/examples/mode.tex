\begin{example}\label{ex:mode}
    Let us return to our vehicale dataset from Example~\ref{ex:dissim}. Using 
    Theorem~\ref{thm:1}, we can identify a mode of our set by taking the most 
    commonly occurring value for each attribute. We can then take these values
    as a vector and call it \(\mu\). In this case, we have:

    \[ 
 	 \mu = \left[\text{H}, \ \text{M}, \ \text{2}, \ \text{5}, \ \text{4}, \ \text{0}\right] 
\]

    This point actually appears in our dataset and corresponds to the first row.
    It is easily verified (a Python script doing so is given as an Appendix)
    that this point is in fact the only point in the span of the attribute space
    \(A_1 \times \cdots \times A_6\) that minimises our summed dissimilarity. In
    this way, we have that the first row of our dataset is the only true mode of
    our set, virtual or not.
\end{example}

