\section{Conclusion}\label{sec:conclusion}

In this paper a novel initialisation method for the \(k\)-modes was introduced
that built on the method set out in the seminal paper~\cite{Huang1998}. The new
method models the final `replacement' process in the original as an instance of
the Hospital-Resident Assignment Problem that may be solved to be mathematically
fair and stable.

Following a thorough description of the \(k\)-modes algorithm and the
established initialisation methods, a comparative analysis was conducted amongst
the three initialisations using both benchmark and artificial datasets. This
analysis revealed that the proposed initialisation was able to outperform both
of the other methods when the choice of \(k\) was optimised according to a
mathematically rigorous elbow method. However, the proposed method was unable to
beat Cao's method (established in~\cite{Cao2009}) when an external framework was
imposed on each dataset by choosing \(k\) to be the number of classes present.

The proposed method should be employed over Cao's when there are no hard
restrictions on what \(k\) may be, or if there is no immediate evidence that the
dataset at hand has some notion of high density. Otherwise, Cao's method remains
the most reliable initialisation in terms of computational time and final cost.
