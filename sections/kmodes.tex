\section{The \(k\)-modes algorithm}\label{sec:kmodes}

The following notation will be used throughout this work to describe the objects
associated with clustering a dataset:

\begin{itemize}
    \item Let \(\mathcal{A} := A_1 \times \cdots \times A_m\) denote the
        \emph{attribute space}. In this work, only categorical attributes are
        considered and so it is intuitive to describe each attribute as a set of
        its values, i.e.\ for each \(j = 1, \ldots, m\) it follows that \(A_j :=
        \left\{a_1^{(j)}, \ldots, a_{d_j}^{(j)}\right\}\) where \(d_j = |A_j|\)
        is considered the size of the \(j^{th}\) attribute.

    \item Let \(\mathcal{X} := \left\{X^{(1)}, \ldots, X^{(N)}\right\} \subset
        \mathcal{A}\) denote a \emph{dataset} where each \(X^{(i)} \in
        \mathcal{X}\) is defined as an \(m\)-tuple \(X^{(i)} := \left(x_1^{(i)},
        \ldots, x_m^{(i)}\right)\) where \(x_j^{(i)} \in A_j\) for each \(j = 1,
        \ldots, m\). The elements of \(\mathcal{X}\) are referred to as
        \emph{data points} or \emph{instances}. A dataset \(\mathcal{X}\) can be
        represented as a table like so:
        \begin{table}[H]
        \centering
        \begin{tabular}{cccccc}
            {} & \(A_1\) & \(A_2\) & \quad \ldots \quad & \(A_{m-1}\) & \(A_m\)
            \\
            \midrule
            \(X^{(1)}\) & \(x_1^{(1)}\) & \(x_2^{(1)}\) & \quad \ldots \quad & 
            \(x_{m-1}^{(1)}\) & \(x_m^{(1)}\)
            \\
            \(X^{(2)}\) & \(x_1^{(2)}\) & \(x_2^{(2)}\) & \quad \ldots \quad &
            \(x_{m-1}^{(2)}\) & \(x_m^{(2)}\)
            \\
            \vdots & \vdots & \vdots & {} & \vdots & \vdots
            \\
            \(X^{(N)}\) & \(x_1^{(N)}\) & \(x_2^{(N)}\) & \quad \ldots \quad &
            \(x_{m-1}^{(N)}\) & \(x_m^{(N)}\)
        \end{tabular}
        \end{table}

    \item Let \(\mathcal{Z} := \left(Z_1, \ldots, Z_k\right)\) be a partition
        of a dataset \(\mathcal{X}\) into \(k \in \mathbb{Z}^{+}\) distinct,
        non-empty parts. Such a partition \(\mathcal{Z}\) is called a
        \emph{clustering} of \(\mathcal{X}\).

    \item Each cluster \(Z_l\) has associated with it a \emph{representative
        point} (see Definition~\ref{def:mode}) which is denoted by \(z^{(l)} =
        \left(z_1^{(l)},~\ldots,~z_m^{(l)}\right) \in \mathcal{A}\).  These
        points may also be referred to as cluster modes. The set of all current
        representative points is denoted \(\overline Z = \left\{z^{(1)}, \ldots,
        z^{(k)}\right\}\).
\end{itemize}


An immediate difference between the \(k\)-means and \(k\)-modes algorithms is 
that they deal with different types of data, and so the metric used to define 
the distance between two points in our space must be different. With 
\(k\)-means, where the data has all-numeric attributes, Euclidean distance is 
often used. However, we do not have this sense of distance with categorical 
data. Instead, we utilise a dissimilarity measure \-- defined below \-- as our 
metric. It can be easily checked that this is indeed a distance measure.

\begin{definition}\label{def:dissim}
    Let \(\mathcal{X}\) be a dataset and consider any \(X^{(a)}, X^{(b)} \in
    \mathcal{X}\). The dissimilarity between \(X^{(a)}\) and \(X^{(b)}\),
    denoted by \(d\left(X^{(a)}, X^{(b)}\right)\), is given by:
    \begin{equation}
        d(X^{(a)}, X^{(b)}) := \sum_{j=1}^{m} \delta(x_j^{(a)}, x_j^{(b)}) \quad
        \text{where} \quad \delta(x, y) = \begin{cases}
                                              0, & \text{if} \ x = y \\
			    		                      1, & \text{otherwise.}
				    	                  \end{cases}
    \end{equation}
    In other words, the dissimilarity between two points is the number of
    attributes where their values are not the same.
\end{definition}

%\begin{example}\label{ex:dissim}
    Throughout this work, we will make use of a number of small examples to aid
    our understanding of various concepts. These examples will utilise a small,
    artificial dataset describing some qualities about vehicles.
    
    The dataset is made up of \(N = 10\) instances, each of which describe a
    vehicle. These instances are defined by \(m = 6\) attributes, the first two
    of which are ordinal variables taken from the set \(\{\text{L, M, H, V}\}\)
    standing for `low', `medium', `high', and `very high' respectively. The
    next three are integer variables and so can be considered as categorical,
    and the final attribute is a binary variable indicating whether the vehicle
    is eco-friendly \((1)\) or not \((0)\). The full dataset is given in
    Table~\ref{tab:dataset}. Please note that there is an additional, unheaded
    column on the left hand side showing the index starting at \(1\) and going
    up to \(5\).
    
    \begin{table}[H]
        \centering
        \singlespacing{%
        \resizebox{.8\textwidth}{!}{%
            \centering
            \begin{tabular}{ccccccc}
\toprule
{} & Buying price & Maintenance costs & No. doors & No. passengers & No. wheels & Eco-friendly \\
\midrule
1  &            H &                 M &         2 &              2 &          4 &            0 \\
2  &            L &                 M &         0 &              1 &          2 &            0 \\
3  &            V &                 H &         2 &              3 &          8 &            0 \\
4  &            H &                 L &         4 &              5 &          4 &            1 \\
5  &            M &                 M &         2 &              5 &          4 &            1 \\
6  &            M &                 L &         2 &              4 &          4 &            1 \\
7  &            V &                 H &         4 &              5 &          4 &            0 \\
8  &            L &                 V &         2 &              4 &          4 &            0 \\
9  &            H &                 M &         0 &              2 &          2 &            1 \\
10 &            H &                 M &         4 &              7 &          4 &            0 \\
\bottomrule
\end{tabular}

        }}
        \caption{The vehicle dataset.}\label{tab:dataset}
    \end{table}

    Let us consider our first two datapoints. With the notation laid out in
    Section~\ref{subsec:notation}, we can express these points as vectors in the
    following way:

    \begin{equation}
        \nonumber
        \begin{aligned}
            X^{(1)} & = & \left[ x_1^{(1)} = \text{H}, \ x_2^{(1)} = \text{M}, \
            x_3^{(1)} = 2, \ x_4^{(1)} = 2, \ x_5^{(1)} = 4, \ x_6^{(1)} = 0 
            \right]
            \\
            X^{(2)} & = & \left[ x_1^{(2)} = \text{L}, \ x_2^{(2)} = \text{M}, \
            x_3^{(2)} = 0, \ x_4^{(2)} = 1, \ x_5^{(2)} = 2, \ x_6^{(2)} = 0
            \right]
        \end{aligned}
    \end{equation}

    Then, by Definition~\ref{def:dissim}, their pairwise dissimilarity is:
    \begin{equation}
        \nonumber
        \begin{aligned}
            \centering
            d(X^{(1)}, X^{(2)}) & = & \delta(\text{H}, \text{L}) & + & 
            \delta(\text{M}, \text{M}) & + & \delta(3, 0) & + & \delta(2, 1) &
            + & \delta(4, 2) & + & \delta(0, 0) & {} & {}
            \\
            {} & = & 1 \ & + & 0 \ & + & 1 \ & + & 1 \ & + & 1 \ & + & 0 \ & = &
            4
        \end{aligned}
    \end{equation}
\end{example}


Now that we have defined a metric on our space, we can turn our attention to
what we mean by the representative point \(\mu^{(l)}\) of a cluster \(C_l\). In
\(k\)-means, these representative points are called centroids or cluster centers
and are defined to be the average of all points \(X^{(i)} \in C_l\). With
categorical data, we use our revised distance measure from
Definition~\ref{def:dissim} to specify a representative point. We typically call
these points a mode of \textbf{X}.

\begin{definition}\label{def:mode}
    Let \(\mathcal{X} \subset \mathcal{A}\) be a dataset and consider some point
    \(z = \left(z_1, \ldots, z_m\right) \in \mathcal{A}\). Then \(z\) is called
    a \emph{mode} of \(\mathcal{X}\) if it minimises the following:
    \begin{equation}\label{eq:summed-dissim}
        D(\mathcal{X}, z) = \sum_{i=1}^{N} d\left(X^{(i)}, z\right)
    \end{equation}
\end{definition}

\begin{definition}\label{def:rel-freq}
    Let \(\mathcal{X} \subset \mathcal{A}\) be a dataset. Then \(n(a_s^{(j)})\)
    denotes the \emph{frequency} of the \(s^{th}\) category
    \(a_s^{(j)}\) of \(A_j\) in \(\mathcal{X}\), i.e.\ for each \(A_j \in
    \mathcal{A}\) and each \(s = 1, \ldots, d_j\):
    \begin{equation}
        n(a_s^{(j)}) := \abs*{%
            {\left\{X^{(i)} \in \mathcal{X}: x_j^{(i)} = a_s^{(j)}\right\}}
        }
    \end{equation}
	
    Furthermore, \(\frac{n(a_s^{(j)})}{N}\) is called the \emph{relative
    frequency} of category \(a_s^{(j)}\) in \textbf{X}.
\end{definition}

\begin{theorem}\label{thm:1}
    Consider a dataset \(\mathcal{X} \subset \mathcal{A}\) and some \(U = (u_1,
    \ldots, u_m) \in \mathcal{A}\). Then \(D(\mathcal{X}, U)\) is minimised if
    and only if \(n(u_j) \geq n(a_s^{(j)})\) for all \(s=1, \ldots, d_j\) for
    each \(j = 1, \ldots, m\).

    A proof of this theorem can be found in the Appendix of~\cite{Huang1998}.
\end{theorem}

%\begin{example}\label{ex:mode}
    Let us return to our vehicale dataset from Example~\ref{ex:dissim}. Using 
    Theorem~\ref{thm:1}, we can identify a mode of our set by taking the most 
    commonly occurring value for each attribute. We can then take these values
    as a vector and call it \(\mu\). In this case, we have:

    \[ 
 	 \mu = \left[\text{H}, \ \text{M}, \ \text{2}, \ \text{5}, \ \text{4}, \ \text{0}\right] 
\]

    This point actually appears in our dataset and corresponds to the first row.
    It is easily verified (a Python script doing so is given as an Appendix)
    that this point is in fact the only point in the span of the attribute space
    \(A_1 \times \cdots \times A_6\) that minimises our summed dissimilarity. In
    this way, we have that the first row of our dataset is the only true mode of
    our set, virtual or not.
\end{example}



\begin{definition}\label{def:cost}
    Let \(\mathcal{Z} = \left\{Z_1, \ldots, Z_k\right\}\) be a clustering of a
    dataset \(\mathcal{X}\), and let \(\overline Z = \left\{z^{(1)},
    \ldots, z^{(k)}\right\}\) be the corresponding cluster modes. Then \(W =
    (w_{i, l})\) is an \(N \times k\) \emph{partition matrix} of \(\mathcal{X}\)
    such that:
    \[
        w_{i, l} = \begin{cases}
                     1, & \text{if} \ X^{(i)} \in Z_l\\
                     0, & \text{otherwise.}
                   \end{cases}
    \]

    With this, the \emph{cost function} is defined to be the summed
    within-cluster dissimilarity:
    \begin{equation}
        C\left(W, \overline Z\right) = \sum_{l=1}^{k} \sum_{i=1}^{N}
        \sum_{j=1}^{m} w_{i,l} \ \delta\left(x_j^{(i)}, z_j^{(l)}\right)
    \end{equation}
\end{definition}

Below is a practical implementation of the \(k\)-modes
algorithm~\cite{Huang1998}:

\begin{singlespace}
    \balg%
    \caption{The \(k\)-modes algorithm}\label{alg:kmodes}
    \KwIn{a dataset \(\mathcal{X}\), a number of clusters to form \(k\)}
    \KwOut{a clustering \(\mathcal{Z}\) of \(\mathcal{X}\)}

    Select \(k\) initial modes \(z^{(1)}, \ldots, z^{(k)} \in \mathcal{X}\)\;
    \(\overline Z \gets \left\{z^{(1)}, \ldots, z^{(k)}\right\}\)\;
    \(\mathcal{Z} \gets \left(\left\{z^{(1)}\right\}, \ldots,
    \left\{z^{(k)}\right\}\right)\)\;

    \For{\(X^{(i)} \in \mathcal{X}\)}{%
        \(Z_{l^*} \gets \textsc{SelectClosest}\left(X^{(i)}\right)\)\;
        \(Z_{l^*} \gets Z_{l^*} \cup \left\{X^{(i)}\right\}\)\;
        \(\textsc{Update}\left(z^{(l^*)}\right)\)\;
    }

    \Repeat{No point changes cluster}{%
        \For{\(X^{(i)} \in \textbf{X}\)}{%
            Let \(Z_l\) be the cluster \(X^{(i)}\) currently belongs to\;
            \(Z_{l^*} \gets \textsc{SelectClosest}\left(X^{(i)}\right)\)\;
            \If{\(l \neq l^*\)}{%                
                \(Z_{l} \gets Z_{l} \setminus \left\{X^{(i)}\right\}\) and
                \(Z_{l^*} \gets Z_{l^*} \cup \left\{X^{(i)}\right\}\)\;
                \(\textsc{Update}\left(z^{(l)}\right)\) and
                \(\textsc{Update}\left(z^{(l^*)}\right)\)\;
            }
        }
    }
\ealg%

\balg%
    \caption{\textsc{SelectClosest}}\label{alg:select_closest}
    \KwIn{%
        a data point \(X^{(i)}\), a set of current clusters \(mathcal{Z}\) and
        their modes \(\overline Z\)
    }
    \KwOut{the cluster whose mode is closest to the data point \(Z_{l^*}\)}

    Select \(z^{l^*} \in \overline Z\) that minimises:
    \(d\left(X^{(i)}, z_{l^*}\right)\)\;
    Find their associated cluster \(Z_{l^*}\)
\ealg%

\balg%
    \caption{\textsc{Update}}
    \KwIn{an attribute space \(\mathcal{A}\), a mode to update \(z^{(l)}\) and
    its cluster \(Z_l\)}
    \KwOut{an updated mode}

    Find \(z \in \mathcal{A}\) that minimises \(D(Z_l, z)\)\;
    \(z^{(l)} \gets z\)\;
\ealg%

    \begin{algorithm}[H]
    \caption{\textsc{Update}}
    \begin{algorithmic}[0]
        \State{\textbf{Input:} a clustering \(C_1, \ldots, C_k\), and a mode to
        update \(\mu^{(l)}\)}
        \State{\textbf{Output:} an updated mode \(\mu^{l_{\text{new}}}\)}

        \State{Find \(\mu^{(l_{\text{new}})}\) that satisfies: 
        \[
            D(C_l, \mu^{(l_{\text{new}})}) = \min_{\mu \in A_1 \times \cdots
            \times A_m} \left\{D(C_l, \mu)\right\}
        \]
        }
        \State{\(\mu^{(l)} \gets \mu^{(l_{\text{new}})}\)}
    \end{algorithmic}
\end{algorithm}

\end{singlespace}

\begin{remark}
    The processes by which the \(k\) initial modes are selected are detailed in 
    Sections~\ref{sec:init}~\&~\ref{sec:proposed-method}.
\end{remark}
