To give comparative results on the quality of the initialisation processes 
defined in
Sections~\ref{sec:init},~\ref{sec:proposed-method}~\&~\ref{sec:preferences},
four well-known, categorical, labelled datasets \- soybean, mushroom, breast
cancer, and zoo animal \- will be clustered with the \(k\)-modes algorithm with
each of the initialisation processes. Rather than comparing our algorithms with
the typical metrics for classification algorithms (accuracy, precision and
recall) the comparison will be based on four clustering performance measures:
adjusted Rand index, adjusted mutual information, homegeneity and completeness.
Definitions of these metrics are given below. In addition to these, we will also
consider the final cost and number of iterations of the best runs for each
initialisation process. As a general rule, each algorithm will be trained on
approximately two thirds of the respective dataset and tested against the final
third.

\subsection{Performance metrics}\label{subsec:metrics}

In the comparison of \(k\)-modes initialisation
processes~\cite{Huang98}\cite{Cao09}, metrics for gauging the quality of
classification algorithms are typically used. Although we will make use of
labelled datasets in this section, as \(k\)-modes is not a classification
algorithm but one for clustering datasets, we will utilise less
classification-specific metrics. In their place are metrics built around more
general characteristics of a given clustering. In some respects, however, the
metrics defined below have been considered similar to their classification
counterparts.\\

\begin{definition}\label{def:contingency}
    Let \textbf{X} be a given dataset and consider two clusterings of this
    dataset:
    \[
        U = \left\{U_1, \ldots, U_r\right\}
        \ \text{and} \
        V = \left\{V_1, \ldots, V_s\right\}
    \]

    We do not require that \(r = s\). Now we define a \emph{contingency table},
    denoted by \(\left[n_{ij}\right]\), to summarise the similarity between our
    two clusterings. This table has entries given by
    \(n_{ij}~=~|U_i~\cap~V_j|\), and a representation of this table is shown in
    Table~\ref{tab:contingency}.

    \begin{table}[H]
    \centering
    \begin{tabular}{cccccc}
        {} & \(V_1\) & \(V_2\) & \(\cdots\) & \(V_s\) & Total
        \\ \midrule
        \(U_1\) & \(n_{11}\) & \(n_{12}\) & \(\cdots\) & \(n_{1s}\) & \(a_1\)
        \\
        \(U_2\) & \(n_{21}\) & \(n_{22}\) & \(\cdots\) & \(n_{2s}\) & \(a_2\)
        \\
        \(\vdots\) & \(\vdots\) & \(\vdots\) & \(\ddots\) & \(\vdots\) &
        \(\vdots\)
        \\
        \(U_r\) & \(n_{r1}\) & \(n_{r2}\) & \(\cdots\) & \(n_{rs}\) & \(a_r\)
        \\ \midrule
        Total & \(b_1\) & \(b_2\) & \(\cdots\) & \(b_s\) & {}
    \end{tabular}
    \caption{The contingency table for two clusterings \(U, V\) of a dataset
    \textbf{X}.}\label{tab:contingency}
    \end{table}

    In addition to these entries, we have taken the sums of the each row and of
    each column, and denoted them by \(a\)~and~\(b\), respectively. That is:
    \[
        a_i = \sum_{j=1}^s n_{ij} \ \forall i = 1, \ldots, r, \quad \text{and}
        \quad b_j = \sum_{i=1}^r n_{ij} \ \forall j = 1, \ldots, s
    \]\\
\end{definition}

\begin{definition}\label{def:adjusted-rand-index}
    Let \textbf{X} be a dataset, \(U, V\) be two clusterings of the elements of
    \textbf{X}, and \(\left[n_{ij}\right]\) be the corresponding contingency
    table. Then the adjusted Rand index of these clusterings is given by:
    \[
        ARI(U, V) = \frac{\displaystyle{\sum_{ij} {n_{ij}\choose 2} -
        \frac{1}{{N\choose 2}}\left[\sum_i {a_i\choose 2}\sum_j {b_j\choose
        2}\right]}}{\displaystyle{\frac{1}{2} \left[\sum_i {a_i\choose 2} +
        \sum_j{b_j\choose 2}\right] - \frac{1}{{N\choose 2}}\left[\sum_i
        {a_i\choose 2}\sum_j {b_j\choose 2}\right]}}
    \]
\end{definition}

\begin{remark}
    In the case where one of \(U\) or \(V\) is a class label of \textbf{X}, we
    can interpret the adjusted Rand index as being a higher-level classification
    accuracy measure; `higher-level' in the sense that it is invariant of how
    our clustering algorithm labels its clusters. The adjusted Rand index
    considers the relationship between pairs of points in both clusterings of
    the dataset rather than each point individually.\\
    
    Say, without loss of generality, that \(V\) is a class label of \textbf{X}
    and \(U\) is some clustering of \textbf{X}. Then it follows that the
    adjusted Rand index takes value \(1\) if \(U\) and \(V\) are indentical (up
    to a permutation of the class labels), \(0\) if \(U\) is as good as a random
    clustering of \textbf{X}, and can take values less than \(0\) if the
    clustering \(U\) is worse than random.\\
\end{remark}

\begin{definition}\label{def:adjusted-mutual-info}
    Let \textbf{X} be a dataset, \(U, V\) be two clusterings of the elements of
    \textbf{X}, and \(\left[n_{ij}\right]\) be the corresponding contingency
    table. Then we define the following quantities:
    \begin{itemize}
        \item Consider a clustering \(C = \left\{C_1, \ldots, C_k\right\}\) of
            \textbf{X}. Then the \emph{entropy}, denoted by \(H(C)\), which is
            associated with this clustering is defined to be:
            \[
                H(C) = - \sum_{i=1}^k P(C_i)\log_2 P(C_i), \quad \text{where}
                \quad P(C_i) = \frac{\left|C_i\right|}{N}
            \]

            \(H(C)\) is non-negative and only takes the value \(0\) when
            \(k~=~1\), i.e.\ there is only one cluster. We can also interpret a
            class labelling of our dataset as a clustering, and so this too will
            have an entropy associated with it.

        \item Consider our two clusterings \(U, V\). We wish to quantify the
            information shared by both clusterings. This quantity is called the
            \emph{mutual information} between these clusterings, and is defined
            to be:
            \[
                MI(U, V) = \sum_{i=1}^r \sum_{j=1}^s P(U_i, V_j)\log_2
                \frac{P(U_i, V_j)}{P(U_i)P(V_j)}, \quad \text{where} \quad P(U_i,
                V_j) = \frac{\left|U_i \cap V_j\right|}{N}
            \]

            \(MI(U, V)\) is also non-negative and is bounded from above by
            \(H(U)\) and \(H(V)\).

        \item Using a combinatorial approach (in a similar fashion to
            Definition~\ref{def:adjusted-rand-index}), and a hypergeometric
            model of randomness, we define the
            \emph{expected~mutual~information} of two random clusterings to be:
            \[
                \mathbb{E}\left(MI(U, V)\right) = \sum_{i=1}^r \sum_{j=1}^s
                \sum_{n_{ij}=n_{ij}^*}^{\min(a_i, b_j)} \frac{n_{ij}}{N}\log_2
                \left(\frac{n_{ij} N}{a_i b_j}\right) \times \frac{a_i!\, b_j!\,
                \left(N - a_i\right)!\, \left(N - b_j\right)!}{N!\, n_{ij}!\,
                \left(a_i - n_{ij}\right)!\, \left(b_j - n_{ij}\right)!\,
                \left(N - a_i - b_j + n_{ij}\right)!}
            \]
            Here, we have \(n_{ij}^* = \max(1, a_i + b_j - N)\).\\
    \end{itemize}
    
    With these quantities, we define the \emph{adjusted mutual information} of
    our pair of clusterings to be:
    \[
        AMI(U, V) = \frac{MI(U, V) - \mathbb{E}\left(MI(U, V)\right)}{\max
        \left(H(U), H(V)\right) - \mathbb{E}\left(MI(U, V)\right)}
    \]
\end{definition}

\begin{remark}
    This adjusted quantity takes values in the interval \(\left[0, 1\right]\).
    In fact, \(AMI(U, V)~=~1\) when \(U\) and \(V\) are indentical, and
    \(AMI(U, V)~=~0\) when the mutual information between the two clusterings is
    equal to the expectation of the mutual information (i.e.\ two random
    cluserings of the data), by chance alone.\\
\end{remark}

\begin{definition}\label{def:homogeneity}
    Let \textbf{X} be a dataset with class labelling \(V\), let \(U\) be a
    clustering of \textbf{X} and let \(\left[n_{ij}\right]\) be the
    corresponding contingency table. The \emph{homogeneity} of \(U\) is defined
    to be:
    \[
        Hom(U) = \begin{cases}
                    \frac{MI(U, V)}{H(V)}, & \ \text{if} \ U \neq V \\
                    1, & \ \text{otherwise}
                 \end{cases}
    \]
    
    Here, \(MI(U, V)\) is the mutual information shared between \(U\) and \(V\),
    and \(H(V)\) is the entropy of the class labelling \(V\).\\

    We interpret the homogeneity of a clustering to be a measure of how `pure'
    each of its cluster are. That is, a clustering satisfies homogeneity if each
    of its clusters only contain points of a single class.\\
\end{definition}

\begin{definition}\label{def:completeness}
    Let \textbf{X} be a dataset with class labelling \(V\), let \(U\) be a
    clustering of \textbf{X} and let \(\left[n_{ij}\right]\) be the
    corresponding contingency table. The \emph{completeness} of \(U\) is defined
    to be:
    \[
        Com(U) = \begin{cases}
                    \frac{MI(U, V)}{H(U)}, & \ \text{if} \ U \neq V \\
                    1, & \ \text{otherwise}
                 \end{cases}
    \]
    
    Here, \(MI(U, V)\) is the mutual information shared between \(U\) and \(V\),
    and \(H(U)\) is the entropy of our clustering, \(U\).\\

    We say that a clustering is `complete' if all the elements of each class
    from \textbf{X} are in the same clusters in \(U\).
\end{definition}

\subsection{The datasets}\label{subsec:datasets}

A bit on the structure of each dataset and links to access them.


\subsection{Results}\label{subsec:results}

Tables of results for each dataset and each initialisation process. Credit to 
\url{https://github.com/nicodv/kmodes} for the Python implementation of both the
Huang and Cao processes, as well as the $k$-modes algorithm itself.

\begin{table}[H]
\resizebox{\textwidth}{!}{%
\centering
    \begin{tabular}{llrrrrrr}
\toprule
    &      &  RandIndex &  Mutual Information &  Homogeneity &  Completeness &  Final Cost &  Number of Iterations \\
Initialisation & {} &            &                     &              &               &             &                       \\
\midrule
Cao & mean &   0.732680 &            0.736883 &     0.839891 &      0.833527 &  159.800000 &              2.100000 \\
    & std &   0.176722 &            0.114551 &     0.058968 &      0.090753 &    5.827140 &              0.316228 \\
Huang & mean &   0.673213 &            0.723988 &     0.891716 &      0.810137 &  152.100000 &              2.300000 \\
    & std &   0.149177 &            0.090000 &     0.040811 &      0.069407 &    3.860052 &              0.527046 \\
Random & mean &   0.679415 &            0.718514 &     0.884622 &      0.806282 &  153.800000 &              2.500000 \\
    & std &   0.104628 &            0.060502 &     0.043801 &      0.042463 &    5.337498 &              0.699206 \\
Matching & mean &   0.685331 &            0.727037 &     0.870267 &      0.814538 &  152.900000 &              2.700000 \\
    & std &   0.131183 &            0.075638 &     0.039603 &      0.059881 &    6.092801 &              0.707107 \\
\bottomrule
\end{tabular}

}
\end{table}
