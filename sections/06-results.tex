To give comparative results on the quality of the initialisation processes defined in Sections \ref{section:init} \& \ref{section:new-method}, four well-known, categorical, labelled datasets - soybean, mushroom, breast cancer, and zoo - will be clustered with the $k$-modes algorithm. Then the typical performance measures of accuracy, precision, and recall will be calculated and summarised below. As a general rule, each algorithm will be trained on approximately two thirds of the respective dataset and tested against the final third.

\begin{definition}
	Let a dataset \textbf{X} have $k$ classes $C_1, \ldots, C_k$, let the number of objects correctly assigned to $C_i$ be denoted $tp_i$, let $fp_i$ denote the number of objects incorrectly assigned to $C_i$, and let $fn_i$ denote the number of objects incorrectly not assigned to $C_i$. Then our performance measures are defined as follows: \\
		
		\centering
		\begin{tabular}{ccc}
			$\emph{Accuracy}: \ \ \frac{\sum_{i=1}^{k}{tp_i}}{|\textbf{X}|}$, &
			
			$\emph{Precision}: \ \ \frac{\sum_{i=1}^{k} \frac{tp_i}{tp_i + fp_i}}{k}$, &
			
			$\emph{Recall}: \ \ \frac{\sum_{i=1}^{k} \frac{tp_i}{tp_i + fn_i}}{k}$ \\
		\end{tabular}
\end{definition}


\subsection{The datasets}\label{subsection:datasets}

A bit on the structure of each dataset and links to access them.


\subsection{Results}\label{subsection:results}

Tables of results for each dataset and each initialisation process. Credit to \url{https://github.com/nicodv/kmodes} for the Python implementation of both the Huang and Cao processes, as well as the $k$-modes algorithm itself.

\begin{example}
\begin{figure}
    \centering
    \begin{tabular}{llrrrrrr}
\toprule
    &      &  RandIndex &  Mutual Information &  Homogeneity &  Completeness &  Final Cost &  Number of Iterations \\
Initialisation & {} &            &                     &              &               &             &                       \\
\midrule
Cao & mean &   0.732680 &            0.736883 &     0.839891 &      0.833527 &  159.800000 &              2.100000 \\
    & std &   0.176722 &            0.114551 &     0.058968 &      0.090753 &    5.827140 &              0.316228 \\
Huang & mean &   0.673213 &            0.723988 &     0.891716 &      0.810137 &  152.100000 &              2.300000 \\
    & std &   0.149177 &            0.090000 &     0.040811 &      0.069407 &    3.860052 &              0.527046 \\
Random & mean &   0.679415 &            0.718514 &     0.884622 &      0.806282 &  153.800000 &              2.500000 \\
    & std &   0.104628 &            0.060502 &     0.043801 &      0.042463 &    5.337498 &              0.699206 \\
Matching & mean &   0.685331 &            0.727037 &     0.870267 &      0.814538 &  152.900000 &              2.700000 \\
    & std &   0.131183 &            0.075638 &     0.039603 &      0.059881 &    6.092801 &              0.707107 \\
\bottomrule
\end{tabular}

\end{figure}
\end{example}

