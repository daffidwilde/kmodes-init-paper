To give comparative results on the quality of the initialisation processes 
defined in
Sections~\ref{sec:init},~\ref{sec:proposed-method}~\&~\ref{sec:preferences},
four well-known, categorical, labelled datasets - soybean, mushroom, breast
cancer, and zoo animal - will be clustered with the \(k\)-modes algorithm with
each of the initialisation processes. Rather than comparing our algorithms with
the typical metrics for classification algorithms (accuracy, precision and
recall) the comparison will be based on four clustering performance measures:
adjusted Rand index, mutual information, homegeneity and completeness.
Definitions of these metrics are given below. In addition to these, we will also
consider the final cost and number of iterations of the best runs for each
initialisation process. As a general rule, each algorithm will be trained on
approximately two thirds of the respective dataset and tested against the final
third.

\subsection{Performance metrics}\label{subsec:metrics}

Usually in the comparison of \(k\)-modes initialisation
processes~\cite{Huang98}\cite{Cao09}, metrics for gauging the quality of
classification algorithms are typically used. As \(k\)-modes is not a
classification algorithm but one for clustering datasets, we will utilise less
class-specific metrics. In their place are metrics built around more general
characteristics of a given clustering. In some respects, however, the metrics
defined below have been considered similar to their classification counterparts.

\begin{definition}\label{def:adjusted-rand}
\end{definition}

\begin{definition}\label{def:mutual-info}
\end{definition}

\begin{definition}\label{def:homogeneity}
\end{definition}

\begin{definition}\label{def:completeness}
\end{definition}

\subsection{The datasets}\label{subsec:datasets}

A bit on the structure of each dataset and links to access them.


\subsection{Results}\label{subsec:results}

Tables of results for each dataset and each initialisation process. Credit to 
\url{https://github.com/nicodv/kmodes} for the Python implementation of both the
Huang and Cao processes, as well as the $k$-modes algorithm itself.

\begin{example}
\begin{figure}
    \resizebox{\textwidth}{!}{%
    \centering
    \begin{tabular}{llrrrrrr}
\toprule
    &      &  RandIndex &  Mutual Information &  Homogeneity &  Completeness &  Final Cost &  Number of Iterations \\
Initialisation & {} &            &                     &              &               &             &                       \\
\midrule
Cao & mean &   0.732680 &            0.736883 &     0.839891 &      0.833527 &  159.800000 &              2.100000 \\
    & std &   0.176722 &            0.114551 &     0.058968 &      0.090753 &    5.827140 &              0.316228 \\
Huang & mean &   0.673213 &            0.723988 &     0.891716 &      0.810137 &  152.100000 &              2.300000 \\
    & std &   0.149177 &            0.090000 &     0.040811 &      0.069407 &    3.860052 &              0.527046 \\
Random & mean &   0.679415 &            0.718514 &     0.884622 &      0.806282 &  153.800000 &              2.500000 \\
    & std &   0.104628 &            0.060502 &     0.043801 &      0.042463 &    5.337498 &              0.699206 \\
Matching & mean &   0.685331 &            0.727037 &     0.870267 &      0.814538 &  152.900000 &              2.700000 \\
    & std &   0.131183 &            0.075638 &     0.039603 &      0.059881 &    6.092801 &              0.707107 \\
\bottomrule
\end{tabular}

}
\end{figure}
\end{example}
