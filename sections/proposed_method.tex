\section{Matching games and the proposed method}\label{sec:method}

Both of the initialisation methods described in Section~\ref{subsec:inits} have
a greedy component. Cao's method essentially chooses the densest point that has
not already been chosen whilst forcing separation between the set of initial
modes. In the case of Huang's, however, the greediness only comes at the end
of the method, when the set of potential modes is replaced by a set of instances
in the dataset. Specifically, this means that in any practical implementation of
this method the order in which a set of potential modes is iterated over can
affect the set of initial modes. Thus, there is no guarantee of consistency.

The initialisation proposed in this work extends Huang's method to be
order-invariant in the final allocation --- thereby eliminating its greedy
component --- and provides a more intuitive starting point for the \(k\)-modes
algorithm. This is done by constructing and solving a matching game between the
set of potential modes and some subset of the data.

In general, matching games are defined by two sets (parties) of players in which
each player creates a preference list of at least some of the players in the
other party. The objective then is to find a `stable' mapping between the two
sets of players such that no pair of players is (rationally) unhappy with their
matching. Algorithms to `solve' --- i.e.\ find stable matchings to --- instances
of matching games are often structured to be party-oriented and aim to maximise
some form of social or party-based optimality~\cite{%
    Erdil2017,Fuku2006,Gale1962,Iwama2016,Kwanashie2015,Manlove2002%
}.

The particular constraints of this case --- where the \(k\) potential modes must
be allocated to a nearby unique data point --- mirror those of the so-called
Hospital-Resident Assignment Problem (HR). This problem gets its name from the
real-world problem of fairly allocating medical students to hospital posts.  A
resident-optimal algorithm for solving HR was presented in~\cite{Gale1962} and
was adapted in~\cite{Roth1984} to take advantage of the structure of the game.
This adapted algorithm is given in Algorithm~\ref{alg:hospital_resident}.

The game used to model HR, its matchings, and its notion of stability are
defined in Definitions~\ref{def:game}---\ref{def:blocking}. A summary of these
definitions in the context of the proposed \(k\)-modes initialisation is given
in Table~\ref{tab:components} before a formal statement of the proposed method
in Algorithm~\ref{alg:proposed_method}.

\begin{definition}\label{def:game}
    Consider two distinct sets \(R, H\) and refer to them residents and
    hospitals. Each \(h \in H\) has a capacity \(c_h \in \mathbb{N}\) associated
    with them. Each player \(r \in R\) and \(h \in H\) has associated 
    with it a strict preference list of the other set's elements such that:
    \begin{itemize}
        \item Each \(r \in R\) ranks a non-empty subset of \(H\), denoted by
            \(f(r)\).
        \item Each \(h \in H\) ranks all and only those residents that have
            ranked it, i.e.\ the preference list of \(h\), denoted \(g(h)\), is
            a permutation of the set
            \(\left\{r \in R \ | \ h \in f(r)\right\}\). If no such residents
            exist, \(h\) is removed from \(H\).
    \end{itemize}

    This construction of residents, hospitals, capacities and preference lists
    is called a \emph{game} and is denoted by \((R, H)\).
\end{definition}

\begin{definition}\label{def:matching}
    Consider a game \((R, H)\). A \emph{matching} \(M\) is any mapping between
    \(R\) and \(H\). If a pair \((r, h) \in R \times H\) are matched in \(M\)
    then this relationship is denoted \(M(r) = h\) and \(r \in M^{-1}(h)\).

    A matching is only considered \emph{valid} if all of the following hold for
    all \(r \in R, h \in H\):
    \begin{itemize}
        \item If \(r\) is matched then \(M(r) \in f(r)\).
        \item If \(h\) has at least one match then \(M^{-1}(h) \subseteq g(h)\).
        \item \(h\) is not over-subscribed, i.e.\ \(\abs*{M^{-1}(h)} \leq c_h\).
    \end{itemize}

    A valid matching is considered \emph{stable} if it does not contain any
    blocking pairs.
\end{definition}

\begin{definition}\label{def:blocking}
    Consider a game \((R, H)\). Then a pair \((r, h) \in R \times H\) is said to
    \emph{block} a matching \(M\) if all of the following hold:
    \begin{itemize}
        \item There is mutual preference, i.e.\ \(r \in g(h)\) and \(h \in
            f(r)\).
        \item Either \(r\) is unmatched or they prefer \(h\) to \(M(r)\).
        \item Either \(h\) is under-subscribed or \(h\) prefers \(r\) to at
            least one resident in \(M^{-1}(h)\).
    \end{itemize}
\end{definition}

\begin{table}[htbp]
    \resizebox{\textwidth}{!}{%
    \begin{tabular}{lcr}
        \toprule%
        Object in \(k\)-modes initialisation & {} & Object in a matching game
        \\\midrule%
        Potential modes & {} & The set of residents
        \\
        Data points closest to potential modes & {} & The set of hospitals
        \\
        Similarity between a potential mode and a point & {} & Respective
        position in each other's preference lists
        \\
        The data point to replace a potential mode & {} & A pair in a matching
        \\\bottomrule
    \end{tabular}}
    \caption{A summary of the relationships between the components of the
             initialisation for \(k\)-modes and those in a matching game
             \((R, H)\).
    }\label{tab:components}
\end{table}

\balg%
    \caption{The hospital-resident algorithm
        (resident-optimal)}\label{alg:hospital_resident}
    \KwIn{a set of residents \(R\), a set of hospitals \(H\), a set of hospital
        capacities \(C\), two preference list functions \(f, g\)}
    \KwOut{a stable, resident-optimal mapping \(M\) between \(R\) and \(H\)}

    \For{\(h \in H\)}{%
        \(M^{-1}(h) \gets \emptyset\)
    }
    \While{There exists any unmatched \(r \in R\) with a non-empty preference
        list}{%
        Take any such resident \(r\) and their most preferred hospital \(h\)\;
        \(\textsc{MatchPair}(s, h)\)\;

        \If{\(\abs*{M^{-1}(h)} > c_h\)}{%
            Find their worst match \(r' \in M^{-1}(h)\)\;
            \(\textsc{UnmatchPair}(r', h)\)\;
        }
        \If{\(\abs*{M^{-1}(h)} = c_h\)}{%
            Find their worst match \(r' \in M^{-1}(h)\)\;
            \For{each successor \(s \in g(h)\) to \(r'\)}{%
                \(\textsc{DeletePair}(s, h)\)
            }
        }
    }
\ealg%

\balg%
    \caption{\textsc{MatchPair}}\label{alg:match}
    \KwIn{a resident \(r\), a hospital \(h\), a matching \(M\)}
    \KwOut{an updated matching \(M\)}

    \(M^{-1}(h) \gets M^{-1}(h) \cup \left\{r\right\}\)\;
\ealg%

\balg%
    \caption{\textsc{UnmatchPair}}\label{alg:unmatch}
    \KwIn{a resident \(r\), a hospital \(h\), a matching \(M\)}
    \KwOut{an updated matching \(M\)}

    \(M^{-1}(h) \gets M^{-1}(h) \setminus \left\{r\right\}\)\;
\ealg%

\balg%
    \caption{\textsc{DeletePair}}\label{alg:delete}
    \KwIn{a resident \(r\), a hospital \(h\)}
    \KwOut{updated preference lists}

    \(f(r) \gets f(r) \setminus \left\{h\right\}\)\;
    \(g(h) \gets g(h) \setminus \left\{r\right\}\)\;
\ealg%

\begin{algorithm}[H]
\caption{The proposed initialisation method}
    \begin{algorithmic}[0]
        \State{\textbf{Input:} a dataset \textbf{X}, with attribute sets \(A_1,
        \ldots, A_m\), and a number of modes to find \(k\)}
        \State{\textbf{Output:} a set of \(k\) initial modes \(\bar{\mu}\)\\}
        \\
        \Comment{Initialisation step}
        \State{\(\tilde{\mu} \gets \emptyset\)}
        \State{\(\bar{\mu} \gets \emptyset\)}
        \State{\(R \gets \emptyset\)}
        \State{\(S \gets \emptyset\)}
        \State{\(C \gets \{c_1, \ldots, c_k\}\)}

        \For{\(j = 1, \ldots, m\)}
            \For{\(s = 1, \ldots, d_j\)}
                \State{Calculate the relative frequency of each attribute value:
                    \(\frac{n(a_s^{(j)})}{N}\).}
	        \EndFor%
        \EndFor%
        \\
        \\
        \Comment{Find the set of virtual modes, \(\tilde{\mu}\), according to
        Huang's method.}
        \For{\(l = 1, \ldots, k\)}
            \For{\(j = 1, \ldots, m\)}
                \State{Consider the probability distribution given by:
                \(
                    \mathbb{P}(A_j) := \left(\frac{n(a_s^{(j)})}{N} : a_s^{(j)}
                    \in A_j\right)
                \)}
                \State{Sample \(a_{s^*}^{(j)}\) from \(A_j\) with respect to
                \(\mathbb{P}(A_j)\).}
                \State{\(\mu_j^{(l)} \gets a_{s^*}^{(j)}\)}
	        \EndFor%
            \State{\(\tilde{\mu} \gets \tilde{\mu} \cup
            \left\{\mu^{(l)}\right\}\)}
	    \EndFor%
        \\
        \\
        \Comment{Construct and solve a capacitated matching game.}
        \State{\(R \gets \tilde{\mu}\)}
        \For{\(r \in R\)}
            \State{\(c_r \gets 1\)}
            \State{Find the set of \(k\) vectors, \(S_r\), in \textbf{X} that 
            are the least dissimilar to \(r\).}
            \State{Arrange \(S_r\) into descending order of similarity.}
            \State{\(S \gets S \cup S_r\)}
        \EndFor%
        \For{\(r \in R\)}
            \State{\(g(r) \gets S_r\)}
        \EndFor%
        \State{Select a method for suitor preference lists and construct
        \(f(s)\) accordingly for each \(s \in S\).}
        \State{Solve the capacitated matching game defined by \((S, R, C)\) to
        obtain a matching \(M:~R~\to~S\).}
        \For{\(r \in R\)}
            \State{\(\bar{\mu} \gets \bar{\mu} \cup \{M(r)\}\)}
        \EndFor%
    \end{algorithmic}
\end{algorithm}

