\section{The proposed method}\label{sec:proposed-method}

Now that we have defined what we mean by a matching game, with the algorithm 
described above, we can construct an alternative initialisation process for the 
\(k\)-modes algorithm.

Let \textbf{X} be a dataset with attribute set \textbf{A}, and let \(\bar{\mu}\) 
be the set of virtual modes found by Huang's method (i.e.\ the set of centroids 
found in the mode which are to be assigned to points in \textbf{X}). We then 
take this set of virtual modes \(\bar{\mu}\) and construct a capacitated 
matching game to be solved by the capacitated Gale-Shapley algorithm in the 
following way.

\begin{singlespace}
    \begin{algorithm}[H]
\caption{The proposed initialisation method}
    \begin{algorithmic}[0]
        \State{\textbf{Input:} a dataset \textbf{X}, with attribute sets \(A_1,
        \ldots, A_m\), and a number of modes to find \(k\)}
        \State{\textbf{Output:} a set of \(k\) initial modes \(\bar{\mu}\)\\}
        \\
        \Comment{Initialisation step}
        \State{\(\tilde{\mu} \gets \emptyset\)}
        \State{\(\bar{\mu} \gets \emptyset\)}
        \State{\(R \gets \emptyset\)}
        \State{\(S \gets \emptyset\)}
        \State{\(C \gets \{c_1, \ldots, c_k\}\)}

        \For{\(j = 1, \ldots, m\)}
            \For{\(s = 1, \ldots, d_j\)}
                \State{Calculate the relative frequency of each attribute value:
                    \(\frac{n(a_s^{(j)})}{N}\).}
	        \EndFor%
        \EndFor%
        \\
        \\
        \Comment{Find the set of virtual modes, \(\tilde{\mu}\), according to
        Huang's method.}
        \For{\(l = 1, \ldots, k\)}
            \For{\(j = 1, \ldots, m\)}
                \State{Consider the probability distribution given by:
                \(
                    \mathbb{P}(A_j) := \left(\frac{n(a_s^{(j)})}{N} : a_s^{(j)}
                    \in A_j\right)
                \)}
                \State{Sample \(a_{s^*}^{(j)}\) from \(A_j\) with respect to
                \(\mathbb{P}(A_j)\).}
                \State{\(\mu_j^{(l)} \gets a_{s^*}^{(j)}\)}
	        \EndFor%
            \State{\(\tilde{\mu} \gets \tilde{\mu} \cup
            \left\{\mu^{(l)}\right\}\)}
	    \EndFor%
        \\
        \\
        \Comment{Construct and solve a capacitated matching game.}
        \State{\(R \gets \tilde{\mu}\)}
        \For{\(r \in R\)}
            \State{\(c_r \gets 1\)}
            \State{Find the set of \(k\) vectors, \(S_r\), in \textbf{X} that 
            are the least dissimilar to \(r\).}
            \State{Arrange \(S_r\) into descending order of similarity.}
            \State{\(S \gets S \cup S_r\)}
        \EndFor%
        \For{\(r \in R\)}
            \State{\(g(r) \gets S_r\)}
        \EndFor%
        \State{Select a method for suitor preference lists and construct
        \(f(s)\) accordingly for each \(s \in S\).}
        \State{Solve the capacitated matching game defined by \((S, R, C)\) to
        obtain a matching \(M:~R~\to~S\).}
        \For{\(r \in R\)}
            \State{\(\bar{\mu} \gets \bar{\mu} \cup \{M(r)\}\)}
        \EndFor%
    \end{algorithmic}
\end{algorithm}

\end{singlespace}

\begin{remark}
    The method for constructing the preference lists of our suitors can affect
    the outcome and performance of this method. Please refer to
    Sections~\ref{sec:preferences}~\&~\ref{sec:results}.
\end{remark}

\begin{example}
    \textcolor{red}{Do this method with the dummy set.}
\end{example}
