\section{The proposed method}\label{sec:proposed-method}

Now that we have defined what we mean by a matching game, with the algorithm 
described above, we can construct an alternative initialisation process for the 
\(k\)-modes algorithm.

Let \textbf{X} be a dataset with attribute set \textbf{A}, and let \(\bar{\mu}\) 
be the set of virtual modes found by Huang's method (i.e.\ the set of centroids 
found in the mode which are to be assigned to points in \textbf{X}). We then 
take this set of virtual modes \(\bar{\mu}\) and construct a capacitated 
matching game to be solved by the capacitated Gale-Shapley algorithm in the 
following way.

\begin{singlespace}
    \balg%
    \caption{The proposed initialisation method}\label{alg:proposed_method}
    \KwIn{a dataset \(\mathcal{X} \subset \mathcal{A}\), a number of modes to
        find \(k\)}
    \KwOut{a set of \(k\) initial modes \(\overline Z\)}

    \(\overline Z \gets \emptyset\)\;
    \(H \gets \emptyset\)\;
    \(R \gets \textsc{SamplePotentialModes}\left(\mathcal{X}\right)\)\;

    \For{\(r \in R\)}{%
        Find the set of \(k\) data points \(H_r \subset \mathcal{X}\) that 
        are the least dissimilar to \(r\)\;
        Arrange \(H_r\) into descending order of similarity with respect to
        \(r\), denoted by \(H_r^*\)\;
        \(H \gets H \cup H_r\)\;
        \(f(r) \gets H_r^*\)\;
    }

    \For{\(h \in H\)}{%
        \(c_h \gets 1\)\;
        Sort \(R\) into descending order of similarity with respect to \(h\),
        denoted by \(R^*\)\;
        \(g(h) \gets R^*\)
    }

    Solve the matching game defined by \((R, H)\) to obtain a matching \(M\)\;
    \For{\(r \in R\)}{%
        \(\overline Z \gets \overline Z \cup \left\{M(r)\right\}\)
    }
\ealg%

\end{singlespace}

\begin{remark}
    The method for constructing the preference lists of our suitors can affect
    the outcome and performance of this method. Please refer to
    Sections~\ref{sec:preferences}~\&~\ref{sec:results}.
\end{remark}

\begin{example}
    \textcolor{red}{Do this method with the dummy set.}
\end{example}
