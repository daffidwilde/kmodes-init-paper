\begin{abstract}
    This paper presents a new way of selecting an initial solution for the
    \(k\)-modes algorithm that allows for a notion of mathematical fairness and
    a leverage of the data that the base initialisations from literature do
    not. The method, which utilises the Hospital-Resident Assignment Problem to
    find the set of initial cluster centroids, is compared with the original
    initialisation for \(k\)-modes as well as the next most popular method. In
    order to highlight the merits of the proposed method two stages of analysis
    are presented. The first uses benchmark datasets to give familiarity whilst
    the second utilises a body of artificial datasets that were generated to
    reveal the nuances of each method. Based on this analysis, the proposed
    method is shown to outperform the other initialisations in the majority of
    cases, especially when the number of clusters is optimised. In addition, the
    proposed method is found to outperform the popular established method
    specifically for low-density data.
\end{abstract}
