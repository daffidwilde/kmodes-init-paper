\begin{abstract}
    This paper presents a new way of selecting an initial solution for the
    \(k\)-modes algorithm that allows for a notion of mathematical fairness and
    a leverage of the data that the base initialisations from literature do not.
    The method, which utilises the Hospital-Resident Assignment Problem to find
    the set of initial cluster centroids, is compared with two initialisation
    methods for \(k\)-modes~\cite{Cao2009} as well as the next most popular
    method present in the literature~\cite{Huang1998}. In order to highlight the
    merits of the proposed method two stages of analysis are presented. The
    paper concludes with an analysis of these methods against the proposed and
    it is demonstrated that the proposed method is able to outperform them both.
    The aim of this analysis is two-fold: first, to highlight the merits of the
    method in a familiar setting by clustering well-known benchmark datasets;
    and second, to provide a deeper insight into how the methods perform against
    one another by generating artificial datasets using the method set out
    in~\cite{Wilde2019}.
\end{abstract}
