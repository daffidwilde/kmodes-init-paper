The \(k\)-modes algorithm is a part of the family of clustering algorithms known
as `prototype-based clustering', and is an extension of the \(k\)-means 
algorithm for categorical data as set out in~\cite{Huang98}. This work will 
outline the key differences and similarities between two established
initialisation methods for the \(k\)-modes algorithm, and in doing so we will be
able to examine how the initial cluster selection process has an impact on the 
efficiency and quality of the \(k\)-modes algorithm. From there, we will define
a new initialisation method that makes use of elements of game theory.\\


\subsection{Notation}\label{subsec:notation}

We will use the following notation throughout this work to describe our dataset,
its points, our clusters and their representative points:

\begin{itemize}
    \item Our dataset has \(N\) elements and is denoted by \textbf{X}.
    \item \textbf{X} is described by a set of \(m \in \mathbb{Z}_+\) attributes 
        \(\textbf{A} = \{A_1, \ldots, A_m\}\).
    \item Each attribute \(A_j\) is considered as a set of attribute values 
        \(A_j = \{a_1^{(j)}, \ldots, a_{d_j}^{(j)}\}\) where \(d_j = |A_j| \in 
        \mathbb{Z}_+\) is sometimes used as shorthand for the size of the 
        \(j^th\) attribute set.
    \item We write each data point \(X^{(i)}, \ i = 1, \ldots, N\), as an 
        \(m\)-dimensional vector:
	    \[
		    X^{(i)} = \left[x_1^{(i)}, x_2^{(i)}, \ldots, x_m^{(i)}\right], \
            \text{where} \ x_j^{(i)} \in A_j \ \text{for all} \ j = 1, \ldots, 
            m.
	    \]
        Here, we refer to \(x_j^{(i)}\) as the value of the \(j^{th}\) attribute
        of the \(i^{th}\) data point, \(X^{(i)}\). A tabular representation is 
        given below:
        \begin{table}[H]
        \centering
        \begin{tabular}{cccccc}
            {} & \(A_1\) & \(A_2\) & \quad \ldots \quad & \(A_{m-1}\) & \(A_m\)
            \\
            \midrule
            \(X^{(1)}\) & \(x_1^{(1)}\) & \(x_2^{(1)}\) & \quad \ldots \quad & 
            \(x_{m-1}^{(1)}\) & \(x_m^{(1)}\)
            \\
            \(X^{(2)}\) & \(x_1^{(2)}\) & \(x_2^{(2)}\) & \quad \ldots \quad &
            \(x_{m-1}^{(2)}\) & \(x_m^{(2)}\)
            \\
            \vdots & \vdots & \vdots & {} & \vdots & \vdots
            \\
            \(X^{(N)}\) & \(x_1^{(N)}\) & \(x_2^{(N)}\) & \quad \ldots \quad &
            \(x_{m-1}^{(N)}\) & \(x_m^{(N)}\)
            \\
        \end{tabular}
        \end{table}
	\item Prototype-based clustering algorithms partition the elements of 
        \(\textbf{X}\) into \(k\) distinct sets (which we call clusters) denoted
        by \(C_1, \ldots, C_k\), where \(k \in \mathbb{Z}_+\) is a 
        pre-determined, fixed integer such that \(k \le N\). That is:
	    \[
		    C_1, \ldots, C_k \ \text{are such that} \ \bigcup_{l=1}^k C_l = 
		    \textbf{X} \quad \text{and} \quad C_l \cap C_t = \emptyset \
		    \text{for all} \ l \neq t
	    \]
    \item Each cluster \(C_l\) has associated with it a representative point 
		(see Section~\ref{subsec:rep-points}) which we denote by 
        \(\mu^{(l)}~=~\left[\mu_1^{(l)},~\ldots,~\mu_m^{(l)}\right]\).
\end{itemize}


\subsection{Dissimilarity measure}\label{subsec:dissim}

An immediate difference between the \(k\)-means and \(k\)-modes algorithms is 
that they deal with different types of data, and so the metric used to define 
the distance between two points in our space must be different. With 
\(k\)-means, where the data has all-numeric attributes, Euclidean distance is 
often used. However, we do not have this sense of distance with categorical 
data. Instead, we utilise a dissimilarity measure - defined below - as our 
metric. It can be easily checked that this is indeed a distance measure.\\

\begin{definition}\label{def:dissim}
    Let \(\textbf{X}\) be a dataset and consider \(X^{(a)}, X^{(b)} \in 
    \textbf{X}\). We define the \emph{dissimilarity} between \(X^{(a)}\) and 
    \(X^{(b)}\), denoted by \(d(X^{(a)}, X^{(b)})\), to be:
	\[
	    d(X^{(a)}, X^{(b)}) = \sum_{j=1}^{m} \delta(x_j^{(a)}, x_j^{(b)}) \quad
	    \text{where} \quad \delta(x, y) = \begin{cases}
                                            0, & \text{if} \ x = y \\
					                        1, & \text{otherwise}
					                      \end{cases}
	\]

    In other words, the dissimilarity between two points is simply the number of
    attributes where their values are not the same. It should also be clear that
    the dissimilarity between a point and itself is always zero.
\end{definition}


\begin{example}\label{ex:dissim}
    Throughout this work, we will make use of a number of small examples to aid 
    our understanding of various concepts. These examples will utilise a small 
    categorical dataset built for the classification of zoo animals (available
    at~\url{https://www.kaggle.com/uciml/zoo-animal-classification/data}).\\

    The dataset is made up of \(N = 101\) instances, each of which describe a
    zoo animal. These instances are defined by \(m = 16\) (mostly binary) 
    attributes, as well as the animal name, and a class label which indicates 
    the type of animal they are. The first five rows of the dataset are given in
    Table~\ref{tab:zoo-head}. Please note that there is an additional, unheaded
    column on the left hand side showing the index starting at \(1\) and going 
    up to \(101\).\\
    
    \begin{table}[H]
    \resizebox{\textwidth}{!}{%
        \centering
        \begin{tabular}{llllllllllllllllllr}
\toprule
{} & animal\_name &   hair &  feathers &   eggs &   milk &  airborne &  aquatic &  predator &  toothed &  backbone &  breathes &  venomous &   fins &  legs &   tail &  domestic &  catsize &  class\_type \\
\midrule
0 &    aardvark &   True &     False &  False &   True &     False &    False &      True &     True &      True &      True &     False &  False &  Four &  False &     False &     True &           1 \\
1 &    antelope &   True &     False &  False &   True &     False &    False &     False &     True &      True &      True &     False &  False &  Four &   True &     False &     True &           1 \\
2 &        bass &  False &     False &   True &  False &     False &     True &      True &     True &      True &     False &     False &   True &  None &   True &     False &    False &           4 \\
3 &        bear &   True &     False &  False &   True &     False &    False &      True &     True &      True &      True &     False &  False &  Four &  False &     False &     True &           1 \\
4 &        boar &   True &     False &  False &   True &     False &    False &      True &     True &      True &      True &     False &  False &  Four &   True &     False &     True &           1 \\
\bottomrule
\end{tabular}

    }
    \caption{The head of the zoo animal dataset}\label{tab:zoo-head}
    \end{table}

    Let us consider our first three data points: aardvark, antelope and bass. 
    Our attributes are given in the central 16 columns (`hair' to `catsize'), 
    and with the notation we laid out in Section~\ref{subsec:notation}, we can 
    express these points as vectors in the following way:

    \begin{equation}
    \nonumber
    \begin{aligned}
        \textbf{Aardvark:} \quad & X^{(1)} = \left[x_1^{(1)} = \true, \ 
            x_2^{(1)} = \false, \ldots, \ x_{15}^{(1)} = \false, \ x_{16}^{(1)} 
            = \true\right]
        \\
        \textbf{Antelope:} \quad & X^{(2)} = \left[x_1^{(2)} = \true, \
            x_2^{(2)} = \false, \ldots, \ x_{15}^{(2)} = \false, \ x_{16}^{(2)} 
            = \true\right]
        \\
        \textbf{Bass:} \quad & X^{(3)} = \left[x_1^{(3)} = \false, \ x_2^{(3)} =
            \false, \ \ldots, \ x_{15}^{(3)} = \false, \ x_{16}^{(3)} = 
            \false\right]
        \\
    \end{aligned}
    \end{equation}

    And using Definition~\ref{def:dissim}, their pairwise dissimilarities are:
    \begin{equation}
    \nonumber
    \begin{aligned}
        d(X^{(1)}, X^{(2)}) & = & 0 + 0 + 0 + 0 + 0 + 0 + 1 + 0 + 0 + 0 + 0 + 0 
        + 0 + 1 + 0 + 0 & = 2
        \\
        d(X^{(1)}, X^{(3)}) & = & 1 + 0 + 1 + 1 + 0 + 1 + 0 + 0 + 0 + 1 + 0 + 1 
        + 1 + 1 + 0 + 1 & = 9
        \\
        d(X^{(2)}, X^{(3)}) & = & 1 + 0 + 1 + 1 + 0 + 1 + 1 + 0 + 0 + 1 + 0 + 1
        + 1 + 0 + 0 + 1 & = 9
        \\
    \end{aligned}
    \end{equation}
    \\
\end{example}

\subsection{Representative points}\label{subsec:rep-points}

Now that we have defined a metric on our space, we can turn our attention to 
what we mean by the representative point \(\mu^{(l)}\) of a cluster \(C_l\). In 
\(k\)-means, we call \(\mu^{(l)}\) a `centroid' and define it to be the average 
of all points \(X^{(i)} \in C_l\) by Euclidean distance. With categorical data, 
we use our revised distance measure from Definition~\ref{def:dissim} to specify 
a representative point. We call such a point a mode of \textbf{X}.\\

\begin{definition}\label{def:mode}
    We define a \emph{mode} of our set \textbf{X} to be any vector \(\mu = 
    [\mu_1, \ldots, \mu_m]\) that minimises the \emph{summed dissimilarity}:
	
    \begin{equation}
        D(\textbf{X}, \mu) = \sum_{i=1}^{N} d(X^{(i)}, \mu)
	\end{equation}
	
    Note that \(\mu\) is not necessarily in \textbf{X}, and in this case we call
    \(\mu\) a \emph{virtual mode} of \textbf{X}.
\end{definition}

\begin{definition}\label{def:rel-freq}
    Let \textbf{X} be a dataset with attributes \(A_1, \ldots, A_m\). Then we
    denote by \(n(a_s^{(j)})\) the \emph{frequency} of the \(s^{th}\) category 
    \(a_s^{(j)}\) of \(A_j\) in \textbf{X}. That is: 
	
    \[
	    n(a_s^{(j)}) := |{\{X^{(i)} \in \textbf{X}: x_j^{(i)} = a_s^{(j)}\}}|
	\]
	
    We call \(\frac{n(a_s^{(j)})}{N}\) the \emph{relative frequency} of category 
    \(a_s^{(j)}\) in \textbf{X}.
\end{definition}

\begin{remark}
    Note that we have \(1 \le n(a_s^{(j)}) \le N\) for all \(s\) and
    \(j~=~1,~\ldots,~m\).\\
\end{remark}

\begin{theorem}\label{thm:1}
    Consider a dataset \textbf{X} and some \(X^{(i)} \in \textbf{X}\). Then:
	
    \[
	    D(\textbf{X}, X^{(i)}) \ \text{is minimised} \ \iff n(x_j^{(i)}) \geq 
	    n(a_s^{(j)}) \ \text{for all} \ s~=~1,~\ldots,~d_j \ \text{for each} \
        j~=~1,~\ldots,~m. 
	\]
\end{theorem}
A proof of this theorem can be found in the Appendix of~\cite{Huang98}.\\

\begin{example}\label{ex:mode}
    Let us return to our zoo animal dataset from Example~\ref{ex:dissim}. Using 
    Theorem~\ref{thm:1}, we can identify a mode of our set by taking the most 
    commonly occurring value for each attribute. This gives us the following 
    vector:
    \[
        \mu = \left[\false, \ \false, \ \true, \ \false, \ \false, \ \false, \
        \true, \ \true, \ \true, \ \true, \ \false, \false, \ \text{Four}, \ 
        \true, \ \false, \ \false\right]
    \]

    In fact, this vector appears in our dataset, and corresponds to the entry 
    for the tuatara:
    
    \begin{table}[H]
        \resizebox{\textwidth}{!}{%
        \centering
        \begin{tabular}{llllllllllllllllllr}
\toprule
{} & animal\_name &   hair &  feathers &  eggs &   milk &  airborne &  aquatic &  predator &  toothed &  backbone &  breathes &  venomous &   fins &  legs &  tail &  domestic &  catsize &  class\_type \\
\midrule
92 &     tuatara &  False &     False &  True &  False &     False &    False &      True &     True &      True &      True &     False &  False &  Four &  True &     False &    False &           3 \\
\bottomrule
\end{tabular}
\label{tab:tuatara}
    }
    \end{table}

    To be consistent with our definition for a mode, we will also calculate our 
    summed dissimilarity, \(D(\textbf{X}, \mu)\). It is easily determined that
    this is \(501\), and that this is the only point in our data which minimises
    the summed dissimilarity. It follows immediately that \(\mu\) is in fact the
    only actual mode of our dataset.
\end{example}

\begin{remark}
    Many practical implementations of the \(k\)-modes algorithm do not actually
    consider \(k\) modes, as there may not be that many points in the dataset
    which minimise the summed dissimilarity. We will still refer to all of the 
    representative points found by the algorithm as modes, however.
\end{remark}

\subsection{The cost function}\label{subsec:cost}

We can use Definitions~\ref{def:dissim}~\&~\ref{def:mode} to determine a cost 
function for our algorithm. When any clustering of the data has been determined,
we can measure the performance of the algorithm against this cost function. Let
our clusters be given by \(C_1, \ldots, C_k\) and our current set of modes be
given by \(\bar{\mu} = \{\mu^{(1)}, \ldots, \mu^{(k)}\}\). Then we define \(W =
(w_{i, l})\) to be an \(N \times k\) partition matrix of \textbf{X} such that:

\[ 
    w_{i,l} = \begin{cases}
                1, & \text{if} \ X^{(i)} \in C_l \\
                0, & \text{otherwise}
              \end{cases}
\]\\

\begin{definition}\label{def:cost}
    For the \(k\)-modes alogrithm, we define our \emph{cost function} to be the 
    summed within-cluster dissimilarity:

    \[
        \text{Cost}(W, \bar{\mu}) = \sum_{l=1}^{k} \sum_{i=1}^{N} \sum_{j=1}^{m}
        w_{i,l} \delta(x_{i,j}, \mu_{l,j})
    \]
\end{definition}


\subsection{The \(k\)-modes algorithm}\label{subsec:kmodes}

Below is a practical implementation of the \(k\)-modes algorithm~\cite{Huang98}:

\balg%
    \caption{The \(k\)-modes algorithm}\label{alg:kmodes}
    \KwIn{a dataset \(\mathcal{X}\), a number of clusters to form \(k\)}
    \KwOut{a clustering \(\mathcal{Z}\) of \(\mathcal{X}\)}

    Select \(k\) initial modes \(z^{(1)}, \ldots, z^{(k)} \in \mathcal{X}\)\;
    \(\overline Z \gets \left\{z^{(1)}, \ldots, z^{(k)}\right\}\)\;
    \(\mathcal{Z} \gets \left(\left\{z^{(1)}\right\}, \ldots,
    \left\{z^{(k)}\right\}\right)\)\;

    \For{\(X^{(i)} \in \mathcal{X}\)}{%
        \(Z_{l^*} \gets \textsc{SelectClosest}\left(X^{(i)}\right)\)\;
        \(Z_{l^*} \gets Z_{l^*} \cup \left\{X^{(i)}\right\}\)\;
        \(\textsc{Update}\left(z^{(l^*)}\right)\)\;
    }

    \Repeat{No point changes cluster}{%
        \For{\(X^{(i)} \in \textbf{X}\)}{%
            Let \(Z_l\) be the cluster \(X^{(i)}\) currently belongs to\;
            \(Z_{l^*} \gets \textsc{SelectClosest}\left(X^{(i)}\right)\)\;
            \If{\(l \neq l^*\)}{%                
                \(Z_{l} \gets Z_{l} \setminus \left\{X^{(i)}\right\}\) and
                \(Z_{l^*} \gets Z_{l^*} \cup \left\{X^{(i)}\right\}\)\;
                \(\textsc{Update}\left(z^{(l)}\right)\) and
                \(\textsc{Update}\left(z^{(l^*)}\right)\)\;
            }
        }
    }
\ealg%

\balg%
    \caption{\textsc{SelectClosest}}\label{alg:select_closest}
    \KwIn{%
        a data point \(X^{(i)}\), a set of current clusters \(mathcal{Z}\) and
        their modes \(\overline Z\)
    }
    \KwOut{the cluster whose mode is closest to the data point \(Z_{l^*}\)}

    Select \(z^{l^*} \in \overline Z\) that minimises:
    \(d\left(X^{(i)}, z_{l^*}\right)\)\;
    Find their associated cluster \(Z_{l^*}\)
\ealg%

\balg%
    \caption{\textsc{Update}}
    \KwIn{an attribute space \(\mathcal{A}\), a mode to update \(z^{(l)}\) and
    its cluster \(Z_l\)}
    \KwOut{an updated mode}

    Find \(z \in \mathcal{A}\) that minimises \(D(Z_l, z)\)\;
    \(z^{(l)} \gets z\)\;
\ealg%

\begin{algorithm}[H]
    \caption{\textsc{Update}}
    \begin{algorithmic}[0]
        \State{\textbf{Input:} a clustering \(C_1, \ldots, C_k\), and a mode to
        update \(\mu^{(l)}\)}
        \State{\textbf{Output:} an updated mode \(\mu^{l_{\text{new}}}\)}

        \State{Find \(\mu^{(l_{\text{new}})}\) that satisfies: 
        \[
            D(C_l, \mu^{(l_{\text{new}})}) = \min_{\mu \in A_1 \times \cdots
            \times A_m} \left\{D(C_l, \mu)\right\}
        \]
        }
        \State{\(\mu^{(l)} \gets \mu^{(l_{\text{new}})}\)}
    \end{algorithmic}
\end{algorithm}


\begin{remark}
    The processes by which the \(k\) initial modes are selected are detailed in 
    Sections~\ref{sec:init}~\&~\ref{sec:proposed-method}.
\end{remark}

