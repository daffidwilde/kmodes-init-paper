\section{Resident preference lists}\label{sec:preferences}

For the purposes of this piece of work and to demonstrate how preference lists
can be generated, a small number will be defined and used. While these
preference lists do not necessarily hold any mathematical justification for why
they could be appropriate or useful at all, they are the simplest methods
available. In fact, research into this area could prove to be promising as a
means of incorporating prior or expert knowledge into the clustering algorithm.

\begin{definition}\label{def:preferences}
    The three preference list methods to be used will be referred to as `Best',
    `Worst' and `Random' from this point, most notably in
    Section~\ref{sec:results}. Their definitions are somewhat self-explanatory
    but are given below.

    Let \((S, R, C)\) be a capacitated matching game where \(R = \tilde{\mu}\)
    and \(S = \left\{S_r : r \in R\right\}\) as in the proposed method. Then the
    reviewer-suitor preference function \(g\) is well-defined by the proposed
    method. Consider some \(s \in S\). Then their preference function \(f(s)\)
    could be defined in the following way:
    \begin{itemize}
        \item \textbf{Best:} Rank the elements of \(R\) in \emph{ascending}
            order of dissimilarity with respect to \(s\) and set this to be
            \(f(s)\).
        \item \textbf{Worst:} Rank the elements of \(R\) in \emph{descending}
            order of dissimilarity with respect to \(s\) and set this to be
            \(f(s)\).
        \item \textbf{Random:} Take a random permutation of the elements of
            \(R\) and set this to be \(f(s)\).
    \end{itemize}
\end{definition}

\begin{remark}
    It should be noted that `Best' could be considered the greediest approach to
    take since it involves choosing the most preferred option for each of the
    slots available in the list. `Random' should also be expected to perform
    badly on average since it creates an element of stupidity in a method that
    is intended to be an intelligent cluster selection. `Worst' is included to
    observe the effects of deliberately choosing a preference list that goes
    against common sense approaches.

    Note also that each of these preference list operators requires a suitor to
    rank all of the reviewers. A slight modification to some (such as the
    inclusion of a threshold on dissimilarity) could drastically improve their
    performance. Also, it is possible to generate a preference list such that
    the proposed method produces an initial clustering identical to that found
    by Huang's method, though that is not discussed here as it has been
    considered trivial.
\end{remark}
