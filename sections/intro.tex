\section{Introduction}\label{sec:intro}

\begin{itemize}
    \item What is clustering?
    \item What is the \(k\)-means paradigm?
    \item What is categorical data and how is it clustered?
\end{itemize}

\subsection{The \(k\)-modes algorithm}\label{subsec:kmodes}

The following notation will be used throughout this work to describe the objects
associated with clustering a dataset:

\begin{itemize}
    \item Let \(\mathcal{A} := A_1 \times \cdots \times A_m\) denote the
        \emph{attribute~space}. In this work, only categorical attributes are
        considered and so it is intuitive to describe each attribute as a set of
        its values, i.e.\ for each \(j = 1, \ldots, m\) it follows that \(A_j :=
        \left\{a_1^{(j)}, \ldots, a_{d_j}^{(j)}\right\}\) where \(d_j = |A_j|\)
        is considered the size of the \(j^{th}\) attribute.

    \item Let \(\mathcal{X} := \left\{X^{(1)}, \ldots, X^{(N)}\right\} \subset
        \mathcal{A}\) denote a \emph{dataset} where each \(X^{(i)} \in
        \mathcal{X}\) is defined as an \(m\)-tuple \(X^{(i)} := \left(x_1^{(i)},
        \ldots, x_m^{(i)}\right)\) where \(x_j^{(i)} \in A_j\) for each \(j = 1,
        \ldots, m\). The elements of \(\mathcal{X}\) are referred to as
        \emph{data points} or \emph{instances}.
%        A dataset \(\mathcal{X}\) can be
%        represented as a table like so:
%        \begin{table}[H]
%        \centering
%        \begin{tabular}{cccccc}
%            {} & \(A_1\) & \(A_2\) & \quad \ldots \quad & \(A_{m-1}\) & \(A_m\)
%            \\
%            \midrule
%            \(X^{(1)}\) & \(x_1^{(1)}\) & \(x_2^{(1)}\) & \quad \ldots \quad & 
%            \(x_{m-1}^{(1)}\) & \(x_m^{(1)}\)
%            \\
%            \(X^{(2)}\) & \(x_1^{(2)}\) & \(x_2^{(2)}\) & \quad \ldots \quad &
%            \(x_{m-1}^{(2)}\) & \(x_m^{(2)}\)
%            \\
%            \vdots & \vdots & \vdots & {} & \vdots & \vdots
%            \\
%            \(X^{(N)}\) & \(x_1^{(N)}\) & \(x_2^{(N)}\) & \quad \ldots \quad &
%            \(x_{m-1}^{(N)}\) & \(x_m^{(N)}\)
%        \end{tabular}
%        \end{table}

    \item Let \(\mathcal{Z} := \left(Z_1, \ldots, Z_k\right)\) be a partition
        of a dataset \(\mathcal{X}\) into \(k \in \mathbb{Z}^{+}\) distinct,
        non-empty parts. Such a partition \(\mathcal{Z}\) is called a
        \emph{clustering} of \(\mathcal{X}\).

    \item Each cluster \(Z_l\) has associated with it a
        \emph{representative~point} (see Definition~\ref{def:mode}) which is
        denoted by \(z^{(l)} = \left(z_1^{(l)},~\ldots,~z_m^{(l)}\right) \in
        \mathcal{A}\).  These points may also be referred to as cluster modes.
        The set of all current representative points is denoted \(\overline Z =
        \left\{z^{(1)}, \ldots, z^{(k)}\right\}\).
\end{itemize}

As is discussed above, the notion of distance is lost in categorical space, and
especially when that space is even partly nominal. Definition~\ref{def:dissim}
describes a simple dissimilarity measure between categorical data points.

\begin{definition}\label{def:dissim}
    Let \(\mathcal{X}\) be a dataset and consider any \(X^{(a)}, X^{(b)} \in
    \mathcal{X}\). The dissimilarity between \(X^{(a)}\) and \(X^{(b)}\),
    denoted by \(d\left(X^{(a)}, X^{(b)}\right)\), is given by:
    \begin{equation}\label{eq:dissim}
        d\left(X^{(a)}, X^{(b)}\right) := \sum_{j=1}^{m} \delta\left(x_j^{(a)},
        x_j^{(b)}\right) \quad \text{where} \quad \delta\left(x, y\right) =
        \begin{cases}
            0, & \text{if} \ x = y \\
            1, & \text{otherwise.}
        \end{cases}
    \end{equation}
    In other words, the dissimilarity between two points is the number of
    attributes where their values are not the same. A proof
    that~\eqref{eq:dissim} is a valid distance metric is given as an appendix.
\end{definition}

%\begin{example}\label{ex:dissim}
    Throughout this work, we will make use of a number of small examples to aid
    our understanding of various concepts. These examples will utilise a small,
    artificial dataset describing some qualities about vehicles.
    
    The dataset is made up of \(N = 10\) instances, each of which describe a
    vehicle. These instances are defined by \(m = 6\) attributes, the first two
    of which are ordinal variables taken from the set \(\{\text{L, M, H, V}\}\)
    standing for `low', `medium', `high', and `very high' respectively. The
    next three are integer variables and so can be considered as categorical,
    and the final attribute is a binary variable indicating whether the vehicle
    is eco-friendly \((1)\) or not \((0)\). The full dataset is given in
    Table~\ref{tab:dataset}. Please note that there is an additional, unheaded
    column on the left hand side showing the index starting at \(1\) and going
    up to \(5\).
    
    \begin{table}[H]
        \centering
        \singlespacing{%
        \resizebox{.8\textwidth}{!}{%
            \centering
            \begin{tabular}{lllrrrr}
\toprule
{} & Price & Maintenance &  Doors &  Passengers &  Wheels &  Eco-Friendly \\
\midrule
0 &     H &           M &      2 &           2 &       4 &             0 \\
1 &     L &           M &      0 &           1 &       2 &             0 \\
2 &     V &           H &      2 &           3 &       8 &             0 \\
3 &     H &           L &      4 &           5 &       4 &             1 \\
4 &     M &           M &      2 &           5 &       4 &             1 \\
5 &     M &           L &      2 &           4 &       4 &             1 \\
6 &     V &           H &      4 &           5 &       4 &             0 \\
7 &     L &           V &      2 &           4 &       4 &             0 \\
8 &     H &           M &      0 &           2 &       2 &             1 \\
9 &     H &           M &      4 &           7 &       4 &             0 \\
\bottomrule
\end{tabular}

        }}
        \caption{The vehicle dataset.}\label{tab:dataset}
    \end{table}

    Let us consider our first two datapoints. With the notation laid out in
    Section~\ref{subsec:notation}, we can express these points as vectors in the
    following way:

    \begin{equation}
        \nonumber
        \begin{aligned}
            X^{(1)} & = & \left[ x_1^{(1)} = \text{H}, \ x_2^{(1)} = \text{M}, \
            x_3^{(1)} = 2, \ x_4^{(1)} = 2, \ x_5^{(1)} = 4, \ x_6^{(1)} = 0 
            \right]
            \\
            X^{(2)} & = & \left[ x_1^{(2)} = \text{L}, \ x_2^{(2)} = \text{M}, \
            x_3^{(2)} = 0, \ x_4^{(2)} = 1, \ x_5^{(2)} = 2, \ x_6^{(2)} = 0
            \right]
        \end{aligned}
    \end{equation}

    Then, by Definition~\ref{def:dissim}, their pairwise dissimilarity is:
    \begin{equation}
        \nonumber
        \begin{aligned}
            \centering
            d(X^{(1)}, X^{(2)}) & = & \delta(\text{H}, \text{L}) & + & 
            \delta(\text{M}, \text{M}) & + & \delta(3, 0) & + & \delta(2, 1) &
            + & \delta(4, 2) & + & \delta(0, 0) & {} & {}
            \\
            {} & = & 1 \ & + & 0 \ & + & 1 \ & + & 1 \ & + & 1 \ & + & 0 \ & = &
            4
        \end{aligned}
    \end{equation}
\end{example}


With this metric defined, the notion of a representative point within a cluster
can be addressed. When clustering numeric data, a centroid of a cluster is taken
to be the average of the points within the cluster so as to summarise the
information contained within that cluster. With categorical data, however, a
frequency approach is used. This follows from the concept of dissimilarity
where the point that best represents (i.e.\ is closest to) those in a cluster
is one with the most frequent attribute values of the points in the cluster. As
such, a representative point of a cluster is often called a mode. The following
definitions and theorem formally define such a representative point and a means
of finding them.

\begin{definition}\label{def:mode}
    Let \(\mathcal{X} \subset \mathcal{A}\) be a dataset and consider some point
    \(z = \left(z_1, \ldots, z_m\right) \in \mathcal{A}\). Then \(z\) is called
    a \emph{mode} of \(\mathcal{X}\) if it minimises the following:
    \begin{equation}\label{eq:summed-dissim}
        D\left(\mathcal{X}, z\right) = \sum_{i=1}^{N} d\left(X^{(i)}, z\right)
    \end{equation}
\end{definition}

\begin{definition}\label{def:rel-freq}
    Let \(\mathcal{X} \subset \mathcal{A}\) be a dataset. Then
    \(n\left(a_s^{(j)}\right)\) denotes the \emph{frequency} of the \(s^{th}\)
    category \(a_s^{(j)}\) of \(A_j\) in \(\mathcal{X}\), i.e.\ for each \(A_j
    \in \mathcal{A}\) and each \(s = 1, \ldots, d_j\):
    \begin{equation}
        n\left(a_s^{(j)}\right) := \abs*{%
            {\left\{X^{(i)} \in \mathcal{X}: x_j^{(i)} = a_s^{(j)}\right\}}
        }
    \end{equation}
	
    Furthermore, \(\frac{n\left(a_s^{(j)}\right)}{N}\) is called the
    \emph{relative~frequency} of category \(a_s^{(j)}\) in \(\mathcal{X}\).
\end{definition}

\begin{theorem}\label{thm:mode}
    Consider a dataset \(\mathcal{X} \subset \mathcal{A}\) and some \(U = (u_1,
    \ldots, u_m) \in \mathcal{A}\). Then \(D(\mathcal{X}, U)\) is minimised if
    and only if \(n\left(u_j\right) \geq n\left(a_s^{(j)}\right)\) for all
    \(s=1, \ldots, d_j\) for each \(j = 1, \ldots, m\).

    A proof of this theorem can be found in the Appendix of~\cite{Huang1998}.
\end{theorem}

%\begin{example}\label{ex:mode}
    Let us return to our vehicale dataset from Example~\ref{ex:dissim}. Using 
    Theorem~\ref{thm:1}, we can identify a mode of our set by taking the most 
    commonly occurring value for each attribute. We can then take these values
    as a vector and call it \(\mu\). In this case, we have:

    \[ 
 	 \mu = \left[\text{H}, \ \text{M}, \ \text{2}, \ \text{5}, \ \text{4}, \ \text{0}\right] 
\]

    This point actually appears in our dataset and corresponds to the first row.
    It is easily verified (a Python script doing so is given as an Appendix)
    that this point is in fact the only point in the span of the attribute space
    \(A_1 \times \cdots \times A_6\) that minimises our summed dissimilarity. In
    this way, we have that the first row of our dataset is the only true mode of
    our set, virtual or not.
\end{example}



Theorem~\ref{thm:mode} defines the process by which representatives are updated
in \(k\)-modes (see Algorithm~\ref{alg:update}), and so the final component from
the \(k\)-means paradigm to be configured is the objective (cost) function. This
function is defined in Definition~\ref{def:cost}, and following that a practical
statement of the \(k\)-modes algorithm is given in Algorithm~\ref{alg:kmodes} as
set out in~\cite{Huang1998}.

\begin{definition}\label{def:cost}
    Let \(\mathcal{Z} = \left\{Z_1, \ldots, Z_k\right\}\) be a clustering of a
    dataset \(\mathcal{X}\), and let \(\overline Z = \left\{z^{(1)},
    \ldots, z^{(k)}\right\}\) be the corresponding cluster modes. Then \(W =
    \left(w_{i, l}\right)\) is an \(N \times k\) \emph{partition~matrix} of
    \(\mathcal{X}\) such that:
    \[
        w_{i, l} = \begin{cases}
                     1, & \text{if} \ X^{(i)} \in Z_l\\
                     0, & \text{otherwise.}
                   \end{cases}
    \]

    With this, the \emph{cost~function} is defined to be the summed
    within-cluster dissimilarity:
    \begin{equation}
        C\left(W, \overline Z\right) := \sum_{l=1}^{k} \sum_{i=1}^{N}
        \sum_{j=1}^{m} w_{i,l} \ \delta\left(x_j^{(i)}, z_j^{(l)}\right)
    \end{equation}
\end{definition}

\begin{algorithm}[H]
    \caption{\(k\)-modes}\label{alg:kmodes}
	\begin{algorithmic}[0]
        \State{\textbf{Input:} a dataset \textbf{X}, a number of clusters to
        form \(k\)}
        \State{\textbf{Output:} a clustering of the dataset \(C_1, \ldots, 
        C_k\)\\}
        \\
        \Comment{Initialisation step}
        \State{\(\bar{\mu} \gets \emptyset\)}
        \For{\(l \in \{1, \ldots, k\}\)}
            \State{\(C_l \gets \emptyset\)}
		\EndFor
        \State{Select \(k\) initial modes, \(\mu^{(1)}, \ldots, \mu^{(k)} \in
        \textbf{X}\).}
        \State{\(\bar{\mu} \gets \left\{\mu^{(1)}, \ldots, 
            \mu^{(k)}\right\}\)\\}
        \\
        \Comment{First cluster assignment}
        \For{\(X_i \in \textbf{X}\)}
            \State{Select \(l^*\) that satisfies: 
                \[
                    d(X^{(i)}, \mu^{(l^*)}) = \min_{1 \le l \le m} 
                    \left\{d(X^{(i)}, \mu^{(l)})\right\}
                \]
            }
            \State{\(C_{l^*} \gets C_{l^*} \cup \{X^{(i)}\}\)}
            \State{\textsc{Update}\((\mu^{(l^*)})\).}
		\EndFor
        \\
        \\
        \Comment{Continue to reassign poiunts to the most similar cluster until
        no point moves}
        \Repeat{%
            \For{\(X^{(i)} \in \textbf{X}\)}
                \For{\(\mu^{(l)} \in \bar{\mu}\)}
                    \State{Calculate \(d(X^{(i)}, \mu^{(l)})\)}
				\EndFor
                \If{\(d(X^{(i)}, \mu^{(l^*)}) > d(X^{(i)}, \mu^{(l')}) \ 
                \text{for some} \ l' \neq l^*\)}
                    \State{\(C_{l^*} \gets C_{l^*} \setminus \{X^{(i)}\}\)}
                    \State{\(C_{l'} \gets C_{l'} \cup \{X^{(i)}\}\)}
                    \State{\textsc{Update}\((\mu^{(l^*)})\) and 
                    \textsc{Update}\((\mu^{(l')})\).}
				\EndIf
			\EndFor
        }
		\Until{No point changes cluster after a full cycle through \textbf{X}.}
	\end{algorithmic}
\end{algorithm}



\subsection{Initialisation processes}\label{subsec:inits}

All of the methods within the \(k\)-means paradigm are heuristics and as
such their performance is dependent on the quality of their initial
solution. The quality of the initial centroids for a particular dataset is
affected by two components: the metric attached to the attribute space and the
process by which they are chosen.

Following the seminal \(k\)-modes papers~\cite{Huang1997a,Huang1997b,Huang1998},
a number of alternative dissimilarity measures have been implemented to improve
on the simple matching dissimilarity defined in~\eqref{eq:dissim}. The main
drawback of this measure is that it often produces clusters with low
intra-cluster similarity~\cite{Ng2007} and does not take into account any
relationships between attributes or their categories. Other measures have been
designed to be used in a specific context where such relationships may be
considered~\cite{Cao2012,Yu2018,Zhou2016}.

\subsubsection{Huang's method}\label{subsec:huang}

In the standard form of the \(k\)-modes algorithm, the \(k\) initial modes are 
chosen at random from \(\mathcal{X}\). Below is an alternative method of
selecting these modes that forces some diversity between them, as described 
in~\cite{Huang1998}. Here, we consider two sets of modes, \(\tilde{\mu}\) and
\(\bar{\mu}\). The former acts as a placeholder set of modes, whereas the latter
is the set of modes to go on to be used by the \(k\)-modes algorithm.

\balg%
    \caption{Huang's method}\label{alg:huang}
    \KwIn{a dataset \(\mathcal{X} \subset \mathcal{A}\), a number of modes to
    find \(k\)}
    \KwOut{a set of \(k\) initial modes \(\overline Z\)}

    \(\widehat Z \gets \emptyset\)\;
    \(\overline Z \gets \emptyset\)\;
    \For{\(j = 1, \ldots, m\)}{%
        \For{\(s = 1, \ldots, d_j\)}{%
            Calculate \(\frac{n(a_s^{(j)})}{N}\)
        }
    }

    \For{\(l = 1, \ldots, k\)}{%
        Create an empty \(m\)-tuple \(\hat{z}^{(l)}\)\;
        \For{\(j = 1, \ldots, m\)}{%
            Sample \(a_{s^*}^{(j)}\) from \(A_j\) with respect to the 
            relative frequencies of \(A_j\)\;
            \(\hat{z}_j^{(l)} \gets a_{s^*}^{(j)}\)
        }

        \(\widehat Z \gets \widehat Z \cup \left\{\hat{z}^{(l)}\right\}\)
    }

    \For{\(\hat{z} \in \widehat Z\)}{%
        Select \(X^{(i^*)} \in \mathcal{X}\) that satisfies: 
        \[
            \min_{1 \leq i \leq N} \left\{d\left(X^{(i)}, \hat{z}\right) \ | \
            X^{(i^*)} \notin \overline Z\right\}
        \]

        \(\overline Z \gets \overline Z \cup \left\{X^{(i^*)}\right\}\)
    }
\ealg%


In the original statement of Huang's method, the algorithm states that the most
frequent categories should be assigned `equally' to the \(k\) initial modes. How
the categories should be distributed `equally' is not well-defined or easily
seen from the example given. This ambiguity in the definition of Huang's method
means that a probabilistic element must be introduced, and unless seeded
pseudo-random numbers are used, computer-generated results are not necessarily
reproducible.

In this work, as is done in the implementation used to apply the \(k\)-modes
algorithm in Section~\ref{sec:results}, the term `equally' is considered to mean
taking a sample from a probability distribution. This distribution is formed by
the relative frequencies of the attributes' values (defined in
Definition~\ref{def:rel-freq}), as is described in Algorithm~\ref{alg:huang}.

In practice, taking a random sample according to some probability distribution
will lead to variation between runs of this method. As such, when Huang's method
is used to initialise the \(k\)-modes algorithm it is typically run multiple
times and the result with lowest final cost is used.

%\balg%
    \caption{Huang's method}\label{alg:huang}
    \KwIn{a dataset \(\mathcal{X} \subset \mathcal{A}\), a number of modes to
    find \(k\)}
    \KwOut{a set of \(k\) initial modes \(\overline Z\)}

    \(\widehat Z \gets \emptyset\)\;
    \(\overline Z \gets \emptyset\)\;
    \For{\(j = 1, \ldots, m\)}{%
        \For{\(s = 1, \ldots, d_j\)}{%
            Calculate \(\frac{n(a_s^{(j)})}{N}\)
        }
    }

    \For{\(l = 1, \ldots, k\)}{%
        Create an empty \(m\)-tuple \(\hat{z}^{(l)}\)\;
        \For{\(j = 1, \ldots, m\)}{%
            Sample \(a_{s^*}^{(j)}\) from \(A_j\) with respect to the 
            relative frequencies of \(A_j\)\;
            \(\hat{z}_j^{(l)} \gets a_{s^*}^{(j)}\)
        }

        \(\widehat Z \gets \widehat Z \cup \left\{\hat{z}^{(l)}\right\}\)
    }

    \For{\(\hat{z} \in \widehat Z\)}{%
        Select \(X^{(i^*)} \in \mathcal{X}\) that satisfies: 
        \[
            \min_{1 \leq i \leq N} \left\{d\left(X^{(i)}, \hat{z}\right) \ | \
            X^{(i^*)} \notin \overline Z\right\}
        \]

        \(\overline Z \gets \overline Z \cup \left\{X^{(i^*)}\right\}\)
    }
\ealg%


\subsubsection{Cao's method}\label{subsec:cao}

Cao's method selects representative points by the average density of a point in
the dataset. As will be seen in the following definition, this average density 
is in fact the average relative frequency of all the attribute values of that 
point. This method is considered deterministic as there is no probabilistic
element \- unlike Huang's method or a random initialisation. So, we can consider
the results to be largely reproducible, except in the case where a tie must be
broken (see Example~\ref{ex:cao}).

\begin{definition}\label{def:density}	
    Consider a data set \(\mathcal{X}\) with attribute set \(\mathcal{A} = 
    \{A_1, \ldots, A_m\}\). Then the \emph{average~density} of any point 
    \(X_i \in \mathcal{X}\) with respect to \(\mathcal{A}\) is 
    defined~\cite{Cao2009} as:
    \begin{equation}\label{eq:density}
        \text{Dens}\left(X^{(i)}\right) = \frac{%
            \sum_{j=1}^m \text{Dens}_{j}\left(X^{(i)}\right)
        }{m}
        \ \ \text{where} \ \
        \text{Dens}_{j}\left(X^{(i)}\right) = \frac{%
            \abs*{%
                \left\{X^{(t)} \in \mathcal{X} : x_j^{(i)} = x_j^{(t)}\right\}
            }
        }{N}
    \end{equation}

    Observe that:
    \[
        \abs*{\left\{X^{(t)} \in \mathcal{X} : x_j^{(i)} = x_j^{(t)}\right\}}%
        = n\left(x_j^{(i)}\right)%
        = \sum_{t=1}^N \left(1 - \delta\left(x_j^{(i)}, x_j^{(t)}\right)\right)
    \]

    And so, an alternative definition for~(\ref{eq:density}) can be derived:
    \begin{equation}\label{eq:density-alt}
    \begin{aligned}
        \text{Dens}\left(X^{(i)}\right)
        & = \frac{1}{mN} \sum_{j=1}^m \sum_{t=1}^N \left(%
            1 - \delta\left(x_j^{(i)}, x_j^{(t)}\right)
        \right)\\
        & = \frac{1}{mN} \sum_{j=1}^m \sum_{t=1}^N 1%
            - \frac{1}{mN} \sum_{j=1}^m \sum_{t=1}^N
            \delta\left(x_j^{(i)}, x_j^{(t)}\right)\\
        & = \frac{mN}{mN} - \frac{1}{mN} \sum_{t=1}^N
            d\left(X^{(i)}, X^{(t)}\right)\\
        & = 1 - \frac{1}{mN} D\left(\mathcal{X}, X^{(i)}\right)
    \end{aligned}
    \end{equation}
\end{definition}

\begin{remark}
    It is worth noting that for all \(X^{(i)} \in \mathcal{X}\) it follows that
    \(\frac{1}{N} \leq \text{Dens}\left(X^{(i)}\right)
    \leq 1\), since:		
	\begin{itemize}	
        \item If \(n\left(x_j^{(i)}\right) = 1\) for all \(j = 1, \ldots, m\)
            then \(%
                \text{Dens}\left(X^{(i)}\right)
                = \frac{\sum_{j=1}^m \frac{1}{N}}{m}
                = \frac{m}{mN}
                = \frac{1}{N}
            \).
        \item If \(n\left(x_j^{(i)}\right) = N\) for all \(j = 1, \ldots, m\)
            then \(%
                \text{Dens}\left(X^{(i)}\right)
                = \frac{\sum_{j=1}^m 1}{m}
                = \frac{m}{m}
                = 1
            \).
	\end{itemize}
\end{remark}

\begin{algorithm}[H]
\caption{Cao's method}\label{alg:cao}
	\begin{algorithmic}[0]
        \State{\textbf{Input:} a dataset \textbf{X}, with attribute sets \(A_1,
        \ldots, A_m\), and a number of modes to find \(k\)}
        \State{\textbf{Output:} a set of \(k\) initial modes \(\bar{\mu}\)\\}
        \\
        \Comment{Initialisation step}
        \State{\(\bar{\mu} \gets \emptyset\)}
        \For{\(X^{(i)} \in \textbf{X}\)}
            \State{Calculate \(\text{Dens}(X^{(i)})\).}
		\EndFor
        \\
        \\
        \Comment{Select the point with maximal density}
        \State{Select \(X^{(i_1)} \in \textbf{X}\) which satisfies:
        \[
            X^{(i_1)} = \argmax_{1 \leq i \leq N} 
            \left\{\text{Dens}(X^{(i)})\right\}
        \]
        }
        \State{\(\bar{\mu} \gets \bar{\mu} \cup \left\{X^{(i_1)}\right\}\)\\}
        \\
        \Comment{Second point maximises both density and distance from the first
        mode}
        \State{Select \(X^{(i_2)} \in \textbf{X}\) which satisfies: 
		\[
            X^{(i_2)} = \argmax_{1 \leq i \leq N} \left\{\text{Dens}(X^{(i)})
            \times d(X^{(i)}, X^{(i_1)})\right\}
		\]
        }
        \\
        \State{\(\bar{\mu} \gets \bar{\mu} \cup \left\{X^{(i_2)}\right\}\)\\}
        \\
        \Comment{Continue to choose points in this fashion until \(k\) are
        chosen}
        \While{\(|\bar{\mu}| < k\)}
            \State{Select \(X^{(i_3)} \in \textbf{X}\) which satisfies:
			\[
                X^{(i_3)} = \argmax_{1 \leq i \leq N} \left\{\min_{\mu^{(l)} \in
                \bar{\mu}} \left\{\text{Dens}(X^{(i)}) \times d(X^{i}, 
                \mu^{(l)})\right\}\right\}
			\]
            }
            \State{\(\bar{\mu} \gets \bar{\mu} \cup \left\{X^{(i_3)}\right\}\)}
		\EndWhile
	\end{algorithmic}
\end{algorithm}



\begin{remark}
    With this alternative definition, we see \-- since \(m\) and \(N\) are fixed
    positive integers \-- that \(\text{Dens}(X^{(i)})\) is maximised when
    \(D(\mathcal{X}, X^{(i)})\) is minimised. Then by Theorem~\ref{thm:1} we have
    that such an \(X^{(i)}\) with maximal average density in \(\mathcal{X}\)
    with respect to \(\mathcal{A}\) is, in fact, a mode of \(\mathcal{X}\). This
    observation allows us to consider some sense of similarity between Huang and
    Cao's methods, as they seem to be trying to achieve the same objective \--
    if only from opposite ends.
\end{remark}

%\begin{example}\label{ex:cao}
    We will now attempt to find \(3\) initial modes for our vehicle dataset 
    using Cao's method, as we did in Example~\ref{ex:huang}. We begin by 
    calculating the average density of each of our points. We rank these in 
    descending order, and take the point with maximal density as our first 
    initial mode. This ranking is shown in Table~\ref{tab:ranked-density}.

    \begin{table}[H]
        \centering
        \singlespacing{%
        \resizebox{.8\textwidth}{!}{%
            \input{tex/ranked_density_table.tex}
        }}
        \caption{The dataset ranked by average
            density.}\label{tab:ranked-density}
    \end{table}

    So, from Table~\ref{tab:ranked-density}, we see that the first row should be
    taken as our first initial mode, \(\mu^{(1)}\). This is something we should
    expect since it was seen in Example~\ref{ex:mode} that this entry has
    minimal summed dissimilarity, and from Equation~\ref{eq:alt-def} we know
    that this is equivalent to maximising density.
    
    Now, we wish to find the point which has the maximal product of its density
    and its dissimilarity with our first mode. One way of doing this is to 
    calculate the dissimilarity between each point and the mode, append this
    as a column to our table and multiply these two new columns by each other
    to give \(\text{Dens}(X^{(i)}) \times d(\mu^{(1)}, X^{(i)})\) for each \(i =
    1, \ldots, 10\). The entries are then ranked by this product, and the first
    entry is taken as the second mode. By inspecting 
    Table~\ref{tab:ranked-dens-dissim}, we see that there is a tie. In practical
    implementations we can only assume that ties are broken arbitrarily. So, we
    shall take the fourth row as our second initial mode, \(\mu^{(2)}\).

    \begin{table}[H]
        \singlespacing{%
        \resizebox{\textwidth}{!}{%
            \begin{tabular}{cccccccccc}
\toprule
{} & Buying price & Maintenance costs &  No. doors &  No. passengers &  No. wheels &  Eco-friendly &   Density &  Dissimilarity &  Density-dissimilarity \\
\midrule
4  &            H &                 L &          4 &               5 &           4 &             1 &  0.383333 &              4 &               1.533333 \\
7  &            V &                 H &          4 &               5 &           4 &             0 &  0.383333 &              4 &               1.533333 \\
6  &            M &                 L &          2 &               4 &           4 &             1 &  0.366667 &              4 &               1.466667 \\
5  &            M &                 M &          2 &               5 &           4 &             1 &  0.433333 &              3 &               1.300000 \\
2  &            L &                 M &          0 &               1 &           2 &             0 &  0.300000 &              4 &               1.200000 \\
8  &            L &                 V &          2 &               4 &           4 &             0 &  0.383333 &              3 &               1.150000 \\
3  &            V &                 H &          2 &               3 &           8 &             0 &  0.283333 &              4 &               1.133333 \\
9  &            H &                 M &          0 &               2 &           2 &             1 &  0.316667 &              3 &               0.950000 \\
10 &            H &                 M &          4 &               7 &           4 &             0 &  0.433333 &              2 &               0.866667 \\
1  &            H &                 M &          2 &               2 &           4 &             0 &  0.483333 &              0 &               0.000000 \\
\bottomrule
\end{tabular}

        }}
        \caption{A ranking of the dataset by those who have highest
            density-dissimilarity product with the first
            mode.}\label{tab:ranked-dens-dissim}
    \end{table}

    In order to find the final initial mode, \(\mu^{(3)}\), we actually need to
    find a pair \((X^{(i_3)}, \mu^{(m)})\) as is stated in 
    Algorithm~\ref{alg:cao}. In order to do this, and the process would be the
    same for any further modes, we must consider all of our current initial
    modes, the dissimilarity between each point in our dataset and these modes,
    and the density of each point in the dataset. A convenient way of displaying
    all of this information is to construct a density-dissimilarity matrix which
    we denote by \(\mathbb{D}\) and define as follows:
    \begin{itemize}
        \item \(\mathbb{D}\) has \(|\bar{\mu}|\) rows and \(N\) columns, where
            \(|\bar{\mu}|\) is the number of initial modes already selected.
        \item The entries of \(\mathbb{D}\) are given by:
            \[
                \mathbb{D}_{li} = \text{Dens}(X^{(i)}) \times d(X^{(i)},
                \mu^{(l)}) \ \text{for all} \ l = 1, \ldots, |\bar{\mu}| \
                \text{and} \ i = 1, \ldots, N
            \]
    \end{itemize}

    Now, we go through each column and highlight the smallest value. These
    represent which current mode has minimal density-dissimilarity with the
    \(i^{th}\) datapoint (column). Then, we go through the highlighted entries
    and select the column which has the largest value. This column corresponds
    to the next datapoint to be selected as an initial mode. This process is
    shown in Figure~\ref{fig:cao-matrix}.
    
    \begin{figure}[H]
        \centering
        \singlespacing{%
        \begin{minipage}{\textwidth}
            \centering
            \(
            \begin{pmatrix}
                0 & 1.2 & 1.1\dot{3} & 1.5\dot{3} & 1.3 & 1.4\dot{6} &
                1.5\dot{3} & 1.15 & 0.95 & 0.8\dot{6}
                \\
                1.9\dot{3} & 1.8 & 1.7 & 0 & 1.3 & 1.1 & 1.15 & 1.91\dot{6} &
                1.2\dot{6} & 1.3
            \end{pmatrix}
            \)
        \end{minipage}

        \vspace{10pt}

        \begin{minipage}{\textwidth}
            \centering
            \(
            \begin{pmatrix}
                \underline{0} & \underline{1.2} &
                \underline{1.1\dot{3}} & 1.5\dot{3} & \underline{1.3}
                & 1.4\dot{6} & 1.5\dot{3} & \underline{1.15} &
                \underline{0.95} & \underline{0.8\dot{6}}
                \\
                1.9\dot{3} & 1.8 & 1.7 & \underline{0} &
                \underline{1.3} & \underline{1.1} &
                \underline{1.15} & 1.91\dot{6} & 1.2\dot{6} & 1.3
            \end{pmatrix}
            \)
        \end{minipage}

        \vspace{10pt}

        \begin{minipage}{\textwidth}
            \centering
            \(
            \begin{pmatrix}
                \underline{0} & \underline{1.2} & \underline{1.1\dot{3}} &
                1.5\dot{3} & \textcolor{red}{\underline{1.3}} & 1.4\dot{6} &
                1.5\dot{3} & \underline{1.15} & \underline{0.95} &
                \underline{0.8\dot{6}}
                \\
                1.9\dot{3} & 1.8 & 1.7 & \underline{0} &
                \textcolor{red}{\underline{1.3}} & \underline{1.1} &
                \underline{1.15} & 1.91\dot{6} & 1.2\dot{6} & 1.3
            \end{pmatrix}
            \)
        \end{minipage}
        }
        \caption{The stages of selecting the \(l^{th}\) mode with a
        density-dissimilarity matrix, for \(l > 2\). First, the row with smaller
        value is highlighted in each column (underlined here). Then of those
        highlighted entries, the entry with maximal value is selected (shown in
    red). Ties are broken arbitrarily.}\label{fig:cao-matrix}
    \end{figure}

    Therefore, our set of initial modes, \(\bar{\mu}\), correspond to the first,
    fourth and fifth rows of our dataset. That is:
    
    \begin{equation}
\nonumber
\begin{aligned}
\bar{\mu} = \{  & \left[\text{2}, \ \text{0}, \ \text{M}, \ \text{2}, \ \text{H}, \ \text{4}\right], \\  & \left[\text{4}, \ \text{1}, \ \text{L}, \ \text{5}, \ \text{H}, \ \text{4}\right], \\  & \left[\text{2}, \ \text{1}, \ \text{M}, \ \text{5}, \ \text{M}, \ \text{4}\right]\} \\ 
\end{aligned}
\end{equation}
\end{example}

