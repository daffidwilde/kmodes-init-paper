\section{Introduction}\label{sec:intro}

%Clustering is an unsupervised learning technique for discovering intrinsic
%structure within data. There exist many approaches to clustering but perhaps the
%most ubiquitous amongst them is centroid-based clustering. This approach aims to
%maximise summed within-cluster similarity by iterating over the points in a
%dataset and adjusting the current clusters according to some measure for central
%tendency until convergence. A popular algorithm for performing centroid-based
%clustering is the \(k\)-means algorithm. In \(k\)-means, a number of groups to
%identify in a dataset, \(k\), is fixed \emph{a priori} and each cluster has
%associated with it a centroid or representative point calculated as the mean of
%the data points within that cluster. Unfortunately, this is only valid for
%numeric data where the mean of a set is well-defined. Despite this, the paradigm
%in which \(k\)-means clustering exists is of interest as it is fast, scalable,
%easily parallelised, and simple in its design~\cite{Wu2009,Zhao2009}.

This work focusses on \(k\)-modes clustering; an extension to
\(k\)-means that permits the sensible clustering of categorical (i.e.\ ordinal,
nominal or otherwise discrete) data as set out in the seminal works by
Huang~\cite{Huang1997a,Huang1997b,Huang1998}. An alternative, though largely
equivalent, form for \(k\)-modes was presented in~\cite{Chaturvedi2001} but is
not considered in this work. Under this framework, the central tendency measure
used to define a centroid or representative point is the mode, and Euclidean
distance is replaced by a simple matching measure. All of the concepts used to
define the \(k\)-modes algorithm are presented later in this section.

The interest of this paper is in how the performance of the \(k\)-modes
algorithm may be affected. Since the \(k\)-modes algorithm is a heuristic, its
performance is dependent on its initial solution. The quality of the initial
solution is affected by two components: the metric being used and the process by
which the solution is chosen.

%Strictly, introducing a new metric alters the space in which the data exists and
%its effect on the initial solution is not independent of the final solution.
%Having said that, following the seminal \(k\)-modes papers, a number of
%alternative dissimilarity measures have been implemented to improve on the
%simple matching dissimilarity used most regularly. The main drawback of the
%standard measure is that it often produces clusters with low intra-cluster
%similarity~\cite{Ng2007} and does not take into account any relationships
%between attributes or their categories. Other measures have been designed to be
%used in a specific context where such relationships may be
%considered~\cite{Cao2012,Yu2018,Zhou2016}. However, these measures sometimes are
%defined between a point in the dataset and a centroid rather than defining a
%metric for the entire space.

%Instead of adjusting the overall space,
This work considers the process by which an initial solution is found. The
proposed method is an extension to that presented by Huang~\cite{Huang1998} that
generates a game-theoretically fair and stable variant to that generated by
% TODO What do these terms mean here? Nothing until matching section.
Huang's method. The remainder of this paper is structured as follows:
\begin{itemize}
    \item Section~\ref{sec:intro} introduces the \(k\)-modes algorithm and its
        established initialisation methods.
    \item Section~\ref{sec:method} provides a brief overview of
        matching games and their variants before a statement of the proposed
        initialisation method.
    \item Section~\ref{sec:results} presents analyses of the initialisation
        methods on benchmark and new, artificial datasets.
    \item Section~\ref{sec:conclusion} concludes the paper.
\end{itemize}


\subsection{The \(k\)-modes algorithm}\label{subsec:kmodes}

The following notation will be used throughout this work to describe the objects
associated with clustering a dataset:

\begin{itemize}
    \item Let \(\mathcal{A} := A_1 \times \cdots \times A_m\) denote the
        \emph{attribute~space}. In this work, only categorical attributes are
        considered, i.e.\ for each \(j = 1, \ldots, m\) it follows that \(A_j :=
        \left\{a_1^{(j)}, \ldots, a_{d_j}^{(j)}\right\}\) where \(d_j = |A_j|\)
        is the size of the \(j^{th}\) attribute.

    \item Let \(\mathcal{X} := \left\{X^{(1)}, \ldots, X^{(N)}\right\} \subset
        \mathcal{A}\) denote a \emph{dataset} where each \(X^{(i)} \in
        \mathcal{X}\) is defined as an \(m\)-tuple \(X^{(i)} := \left(x_1^{(i)},
        \ldots, x_m^{(i)}\right)\) where \(x_j^{(i)} \in A_j\) for each \(j = 1,
        \ldots, m\). The elements of \(\mathcal{X}\) are referred to as
        \emph{data points} or \emph{instances}.
%        A dataset \(\mathcal{X}\) can be
%        represented as a table like so:
%        \begin{table}[H]
%        \centering
%        \begin{tabular}{cccccc}
%            {} & \(A_1\) & \(A_2\) & \quad \ldots \quad & \(A_{m-1}\) & \(A_m\)
%            \\
%            \midrule
%            \(X^{(1)}\) & \(x_1^{(1)}\) & \(x_2^{(1)}\) & \quad \ldots \quad & 
%            \(x_{m-1}^{(1)}\) & \(x_m^{(1)}\)
%            \\
%            \(X^{(2)}\) & \(x_1^{(2)}\) & \(x_2^{(2)}\) & \quad \ldots \quad &
%            \(x_{m-1}^{(2)}\) & \(x_m^{(2)}\)
%            \\
%            \vdots & \vdots & \vdots & {} & \vdots & \vdots
%            \\
%            \(X^{(N)}\) & \(x_1^{(N)}\) & \(x_2^{(N)}\) & \quad \ldots \quad &
%            \(x_{m-1}^{(N)}\) & \(x_m^{(N)}\)
%        \end{tabular}
%        \end{table}

    \item Let \(\mathcal{Z} := \left(Z_1, \ldots, Z_k\right)\) be a partition
        of a dataset \(\mathcal{X}\) into \(k \in \mathbb{Z}^{+}\) distinct,
        non-empty parts. Such a partition \(\mathcal{Z}\) is called a
        \emph{clustering} of \(\mathcal{X}\).

    \item Each cluster \(Z_l\) has associated with it a
        \emph{representative~point} (see Definition~\ref{def:mode}) which is
        denoted by \(z^{(l)} = \left(z_1^{(l)},~\ldots,~z_m^{(l)}\right) \in
        \mathcal{A}\).  These points may also be referred to as cluster modes.
        The set of all current representative points is denoted \(\overline Z =
        \left\{z^{(1)}, \ldots, z^{(k)}\right\}\).
\end{itemize}

%As is discussed above, the notion of distance is lost in categorical space, and
%especially when that space is even partly nominal.
Definition~\ref{def:dissim} describes a dissimilarity measure between
categorical data points.

\begin{definition}\label{def:dissim}
    Let \(\mathcal{X} \subset \mathcal A\) be a dataset and consider any
    \(X^{(a)}, X^{(b)} \in \mathcal{X}\). The dissimilarity between \(X^{(a)}\)
    and \(X^{(b)}\), denoted by \(d\left(X^{(a)}, X^{(b)}\right)\), is given by:
    \begin{equation}\label{eq:dissim}
        d\left(X^{(a)}, X^{(b)}\right) := \sum_{j=1}^{m} \delta\left(x_j^{(a)},
        x_j^{(b)}\right) \quad \text{where} \quad \delta\left(x, y\right) =
        \begin{cases}
            0, & \text{if} \ x = y \\
            1, & \text{otherwise.}
        \end{cases}
    \end{equation}
    %In other words, the dissimilarity between two points is the number of
    %attributes where their values are not the same. A proof
    %that~\eqref{eq:dissim} is a valid distance metric is given as an appendix.
\end{definition}

%\begin{example}\label{ex:dissim}
    Throughout this work, we will make use of a number of small examples to aid
    our understanding of various concepts. These examples will utilise a small,
    artificial dataset describing some qualities about vehicles.
    
    The dataset is made up of \(N = 10\) instances, each of which describe a
    vehicle. These instances are defined by \(m = 6\) attributes, the first two
    of which are ordinal variables taken from the set \(\{\text{L, M, H, V}\}\)
    standing for `low', `medium', `high', and `very high' respectively. The
    next three are integer variables and so can be considered as categorical,
    and the final attribute is a binary variable indicating whether the vehicle
    is eco-friendly \((1)\) or not \((0)\). The full dataset is given in
    Table~\ref{tab:dataset}. Please note that there is an additional, unheaded
    column on the left hand side showing the index starting at \(1\) and going
    up to \(5\).
    
    \begin{table}[H]
        \centering
        \singlespacing{%
        \resizebox{.8\textwidth}{!}{%
            \centering
            \begin{tabular}{ccccccc}
\toprule
{} & Buying price & Maintenance costs & No. doors & No. passengers & No. wheels & Eco-friendly \\
\midrule
1  &            H &                 M &         2 &              2 &          4 &            0 \\
2  &            L &                 M &         0 &              1 &          2 &            0 \\
3  &            V &                 H &         2 &              3 &          8 &            0 \\
4  &            H &                 L &         4 &              5 &          4 &            1 \\
5  &            M &                 M &         2 &              5 &          4 &            1 \\
6  &            M &                 L &         2 &              4 &          4 &            1 \\
7  &            V &                 H &         4 &              5 &          4 &            0 \\
8  &            L &                 V &         2 &              4 &          4 &            0 \\
9  &            H &                 M &         0 &              2 &          2 &            1 \\
10 &            H &                 M &         4 &              7 &          4 &            0 \\
\bottomrule
\end{tabular}

        }}
        \caption{The vehicle dataset.}\label{tab:dataset}
    \end{table}

    Let us consider our first two datapoints. With the notation laid out in
    Section~\ref{subsec:notation}, we can express these points as vectors in the
    following way:

    \begin{equation}
        \nonumber
        \begin{aligned}
            X^{(1)} & = & \left[ x_1^{(1)} = \text{H}, \ x_2^{(1)} = \text{M}, \
            x_3^{(1)} = 2, \ x_4^{(1)} = 2, \ x_5^{(1)} = 4, \ x_6^{(1)} = 0 
            \right]
            \\
            X^{(2)} & = & \left[ x_1^{(2)} = \text{L}, \ x_2^{(2)} = \text{M}, \
            x_3^{(2)} = 0, \ x_4^{(2)} = 1, \ x_5^{(2)} = 2, \ x_6^{(2)} = 0
            \right]
        \end{aligned}
    \end{equation}

    Then, by Definition~\ref{def:dissim}, their pairwise dissimilarity is:
    \begin{equation}
        \nonumber
        \begin{aligned}
            \centering
            d(X^{(1)}, X^{(2)}) & = & \delta(\text{H}, \text{L}) & + & 
            \delta(\text{M}, \text{M}) & + & \delta(3, 0) & + & \delta(2, 1) &
            + & \delta(4, 2) & + & \delta(0, 0) & {} & {}
            \\
            {} & = & 1 \ & + & 0 \ & + & 1 \ & + & 1 \ & + & 1 \ & + & 0 \ & = &
            4
        \end{aligned}
    \end{equation}
\end{example}


With this metric defined, the notion of a representative point within a cluster
can be addressed. When clustering numeric data, a centroid of a cluster is taken
to be the average of the points within the cluster so as to summarise the
information contained within that cluster. With categorical data, however, a
frequency approach is used. This follows from the concept of dissimilarity
where the point that best represents (i.e.\ is closest to) those in a cluster
is one with the most frequent attribute values of the points in the cluster. As
such, a representative point of a cluster is often called a mode. The following
definitions and theorem formally define such a representative point and a means
of finding them.

\begin{definition}\label{def:mode}
    Let \(\mathcal{X} \subset \mathcal{A}\) be a dataset and consider some point
    \(z = \left(z_1, \ldots, z_m\right) \in \mathcal{A}\). Then \(z\) is called
    a \emph{mode} of \(\mathcal{X}\) if it minimises the following:
    \begin{equation}\label{eq:summed-dissim}
        D\left(\mathcal{X}, z\right) = \sum_{i=1}^{N} d\left(X^{(i)}, z\right)
    \end{equation}
\end{definition}

\begin{definition}\label{def:rel-freq}
    Let \(\mathcal{X} \subset \mathcal{A}\) be a dataset. Then
    \(n\left(a_s^{(j)}\right)\) denotes the \emph{frequency} of the \(s^{th}\)
    category \(a_s^{(j)}\) of \(A_j\) in \(\mathcal{X}\), i.e.\ for each \(A_j
    \in \mathcal{A}\) and each \(s = 1, \ldots, d_j\):
    \begin{equation}
        n\left(a_s^{(j)}\right) := \abs*{%
            {\left\{X^{(i)} \in \mathcal{X}: x_j^{(i)} = a_s^{(j)}\right\}}
        }
    \end{equation}
	
    Furthermore, \(\frac{n\left(a_s^{(j)}\right)}{N}\) is called the
    \emph{relative~frequency} of category \(a_s^{(j)}\) in \(\mathcal{X}\).
\end{definition}

\begin{theorem}\label{thm:mode}
    Consider a dataset \(\mathcal{X} \subset \mathcal{A}\) and some \(U = (u_1,
    \ldots, u_m) \in \mathcal{A}\). Then \(D(\mathcal{X}, U)\) is minimised if
    and only if \(n\left(u_j\right) \geq n\left(a_s^{(j)}\right)\) for all
    \(s=1, \ldots, d_j\) for each \(j = 1, \ldots, m\).

    A proof of this theorem can be found in the Appendix of~\cite{Huang1998}.
\end{theorem}

%\begin{example}\label{ex:mode}
    Let us return to our vehicale dataset from Example~\ref{ex:dissim}. Using 
    Theorem~\ref{thm:1}, we can identify a mode of our set by taking the most 
    commonly occurring value for each attribute. We can then take these values
    as a vector and call it \(\mu\). In this case, we have:

    \[ 
 	 \mu = \left[\text{H}, \ \text{M}, \ \text{2}, \ \text{5}, \ \text{4}, \ \text{0}\right] 
\]

    This point actually appears in our dataset and corresponds to the first row.
    It is easily verified (a Python script doing so is given as an Appendix)
    that this point is in fact the only point in the span of the attribute space
    \(A_1 \times \cdots \times A_6\) that minimises our summed dissimilarity. In
    this way, we have that the first row of our dataset is the only true mode of
    our set, virtual or not.
\end{example}



Theorem~\ref{thm:mode} defines the process by which representatives are updated
in \(k\)-modes (see Algorithm~\ref{alg:update}), and so the final component from
the \(k\)-means paradigm to be configured is the objective (cost) function. This
function is defined in Definition~\ref{def:cost}, and following that a practical
statement of the \(k\)-modes algorithm is given in Algorithm~\ref{alg:kmodes} as
set out in~\cite{Huang1998}.

% TODO Should the term `representative point' be omitted to avoid confusion with
% `mode' later on?
\begin{definition}\label{def:cost}
    Let \(\mathcal{Z} = \left\{Z_1, \ldots, Z_k\right\}\) be a clustering of a
    dataset \(\mathcal{X}\), and let \(\overline Z = \left\{z^{(1)},
    \ldots, z^{(k)}\right\}\) be the corresponding cluster modes. Then \(W =
    \left(w_{i, l}\right)\) is an \(N \times k\) \emph{partition~matrix} of
    \(\mathcal{X}\) such that:
    \[
        w_{i, l} = \begin{cases}
                     1, & \text{if} \ X^{(i)} \in Z_l\\
                     0, & \text{otherwise.}
                   \end{cases}
    \]

    With this, the \emph{cost~function} is defined to be the summed
    within-cluster dissimilarity:
    \begin{equation}\label{eq:cost}
        C\left(W, \overline Z\right) := \sum_{l=1}^{k} \sum_{i=1}^{N}
        \sum_{j=1}^{m} w_{i,l} \ \delta\left(x_j^{(i)}, z_j^{(l)}\right)
    \end{equation}
\end{definition}

\balg%
    \caption{The \(k\)-modes algorithm}\label{alg:kmodes}
    \KwIn{a dataset \(\mathcal{X}\), a number of clusters to form \(k\)}
    \KwOut{a clustering \(\mathcal{Z}\) of \(\mathcal{X}\)}

    Select \(k\) initial modes \(z^{(1)}, \ldots, z^{(k)} \in \mathcal{X}\)\;
    \(\overline Z \gets \left\{z^{(1)}, \ldots, z^{(k)}\right\}\)\;
    \(\mathcal{Z} \gets \left(\left\{z^{(1)}\right\}, \ldots,
    \left\{z^{(k)}\right\}\right)\)\;

    \For{\(X^{(i)} \in \mathcal{X}\)}{%
        \(Z_{l^*} \gets \textsc{SelectClosest}\left(X^{(i)}\right)\)\;
        \(Z_{l^*} \gets Z_{l^*} \cup \left\{X^{(i)}\right\}\)\;
        \(\textsc{Update}\left(z^{(l^*)}\right)\)\;
    }

    \Repeat{No point changes cluster}{%
        \For{\(X^{(i)} \in \textbf{X}\)}{%
            Let \(Z_l\) be the cluster \(X^{(i)}\) currently belongs to\;
            \(Z_{l^*} \gets \textsc{SelectClosest}\left(X^{(i)}\right)\)\;
            \If{\(l \neq l^*\)}{%                
                \(Z_{l} \gets Z_{l} \setminus \left\{X^{(i)}\right\}\) and
                \(Z_{l^*} \gets Z_{l^*} \cup \left\{X^{(i)}\right\}\)\;
                \(\textsc{Update}\left(z^{(l)}\right)\) and
                \(\textsc{Update}\left(z^{(l^*)}\right)\)\;
            }
        }
    }
\ealg%

\balg%
    \caption{\textsc{SelectClosest}}\label{alg:select_closest}
    \KwIn{%
        a data point \(X^{(i)}\), a set of current clusters \(mathcal{Z}\) and
        their modes \(\overline Z\)
    }
    \KwOut{the cluster whose mode is closest to the data point \(Z_{l^*}\)}

    Select \(z^{l^*} \in \overline Z\) that minimises:
    \(d\left(X^{(i)}, z_{l^*}\right)\)\;
    Find their associated cluster \(Z_{l^*}\)
\ealg%

\balg%
    \caption{\textsc{Update}}
    \KwIn{an attribute space \(\mathcal{A}\), a mode to update \(z^{(l)}\) and
    its cluster \(Z_l\)}
    \KwOut{an updated mode}

    Find \(z \in \mathcal{A}\) that minimises \(D(Z_l, z)\)\;
    \(z^{(l)} \gets z\)\;
\ealg%


The standard selection method to initialise \(k\)-modes is to randomly sample
\(k\) distinct points in the dataset. In all cases, the initial modes must be
points in the dataset to ensure that there are no empty clusters in the first
iteration of the algorithm. The remainder of this section describes two
well-established initialisation methods that aim to preemptively lever the
structure of the data at hand.


\subsection{Initialisation processes}\label{subsec:inits}

\subsubsection{Huang's method}\label{subsec:huang}

Amongst the original works by Huang, an alternative initialisation method was
presented that selects modes by distributing frequently occurring values from the
attribute space among \(k\) potential modes~\cite{Huang1998}. The process,
denoted as Huang's method, is described in full in Algorithm~\ref{alg:huang}.
Huang's method considers a set of potential modes,
\(\widehat Z \subset \mathcal A\), that is then replaced by the actual set of
initial modes, \(\overline Z \subset \mathcal X\).

%In the original statement of Huang's method, it is stated that the most
%frequent categories should be assigned `equally' to the set of potential modes.
%How the categories should be distributed `equally' is not well-defined or easily
%seen from the example given in the paper. In software implementations, including
%the one used in Section~\ref{sec:results}, the term is taken to mean using a
%probability distribution to sample values from the attribute space. This
%probability distribution is formed by the relative frequencies of each
%attribute's categories.

\begin{algorithm}[H]
\caption{Huang's method}\label{alg:huang}
    \begin{algorithmic}[0]
        \State{\textbf{Input:} a dataset \textbf{X}, with attribute sets \(A_1,
        \ldots, A_m\), and a number of modes to find \(k\)}
        \State{\textbf{Output:} a set of \(k\) initial modes \(\bar{\mu}\)\\}
        \\
        \Comment{Initialisation step}
        \State{\(\tilde{\mu} \gets \emptyset\)}
        \State{\(\bar{\mu} \gets \emptyset\)}
        \For{\(j = 1, \ldots, m\)}
            \For{\(s = 1, \ldots, d_j\)}
                \State{Calculate the relative frequency of each attribute value:
                    \(\frac{n(a_s^{(j)})}{N}\).}
	        \EndFor
        \EndFor
        \\
        \\
        \Comment{Distribute most common attribute values}
        \For{\(l = 1, \ldots, k\)}
            \For{\(j = 1, \ldots, m\)}
                \State{Sample \(a_{s^*}^{(j)}\) from \(A_j\) by considering the 
                relative frequencies of the elements of \(A_j\) as a probability
                distribution.}
                \State{\(\mu_j^{(l)} \gets a_{s^*}^{(j)}\)}
	        \EndFor
            \State{\(\tilde{\mu} \gets \tilde{\mu} \cup \{\mu^{(l)}\}\)}
	    \EndFor
        \\
        \\
        \Comment{Replace \(\tilde{\mu}\) with points in \textbf{X} to avoid
        empty clusters}
        \For{\(\mu \in \tilde{\mu}\)}
            \State{Select \(X^{(i^*)} \in \textbf{X}\) such that: 
                \[
                    X^{(i^*)} = \argmin_{1 \leq i \leq N} \left\{ d(X^{(i)}, 
                    \mu): \ X^{(i^*)} \neq \mu'  \ \forall \mu' \in 
                    \bar{\mu}\right\}
                \]
            }
            \State{\(\bar{\mu} \gets \bar{\mu} \cup \left\{X^{(i^*)}\right\}\)}
        \EndFor
    \end{algorithmic}
\end{algorithm}

%\begin{example}\label{ex:huang}
    Consider our vehicle dataset. We will now find a set of initial modes for
    the \(k\)-modes algorithm using Huang's method. For the sake of this
    example, we will let \(k = 3\).
    
    \begin{table}[H]
    \centering
    \singlespacing{%
    \resizebox{.8\textwidth}{!}{%
        \begin{tabular}{lrrrrrr}
\toprule
{} &  Doors &  Eco-Friendly &  Maintenance &  Passengers &  Price &  Wheels \\
\midrule
0 &    0.2 &           0.6 &          0.0 &         0.0 &    0.0 &     0.0 \\
1 &    0.0 &           0.4 &          0.0 &         0.1 &    0.0 &     0.0 \\
2 &    0.5 &           0.0 &          0.0 &         0.2 &    0.0 &     0.2 \\
3 &    0.0 &           0.0 &          0.0 &         0.1 &    0.0 &     0.0 \\
4 &    0.3 &           0.0 &          0.0 &         0.2 &    0.0 &     0.7 \\
5 &    0.0 &           0.0 &          0.0 &         0.3 &    0.0 &     0.0 \\
7 &    0.0 &           0.0 &          0.0 &         0.1 &    0.0 &     0.0 \\
8 &    0.0 &           0.0 &          0.0 &         0.0 &    0.0 &     0.1 \\
L &    0.0 &           0.0 &          0.2 &         0.0 &    0.2 &     0.0 \\
M &    0.0 &           0.0 &          0.5 &         0.0 &    0.2 &     0.0 \\
H &    0.0 &           0.0 &          0.2 &         0.0 &    0.4 &     0.0 \\
V &    0.0 &           0.0 &          0.1 &         0.0 &    0.2 &     0.0 \\
\bottomrule
\end{tabular}

    }}
    \caption{Relative frequency table for attribute values.}\label{tab:rel-freq}
    \end{table}

    We begin by calculating the relative frequencies of our attributes' values,
    which are stored in Table~\ref{tab:rel-freq}. Now to find our set of
    (potentially) virtual modes, \(\tilde{\mu}\). For each attribute, we will
    take a sample of size one from the its set of values according to the
    probability distribution represented in the corresponding column of
    Table~\ref{tab:rel-freq}. Let us begin with the first attribute of our first
    mode in \(\tilde{\mu}\). Then we have to sample from the following
    probability distribution:
    
    \begin{table}[H]
    \centering
    \singlespacing{%
    \begin{tabular}{cccccc}
        \(A_{1}\) &\vline& L & M & H & V \\
        \midrule\(\mathbb{P}(A_{1} = a_s^{(1)})\) &\vline& \(\frac{2}{10}\) &
        \(\frac{2}{10}\) & \(\frac{4}{10}\) & \(\frac{2}{10}\) 
    \end{tabular}
    }
    \end{table}
    
    We sample one value from this distribution and set that value to be the
    first component of our first mode. This process is repeated for all values
    \(l = 1, 2, 3\) and \(j = 1, \ldots, 6\) giving us \(3\) \(m\)-dimensional
    vectors that fairly represent the most frequent attribute values. This set
    of vectors is \(\tilde{\mu}\).

    There are many ways of obtaining \(\tilde{\mu}\) from our relative frequency
    table but we have opted to do so using a short Python script (see Appendix).
    In this case, we have the following set of vectors:

    \begin{equation} 
\begin{aligned} 
	\tilde{\mu} = \left\{ & \left[\text{L}, \ \text{M}, \ \text{4}, \ \text{2}, \ \text{4}, \ \text{0}\right], \\ & \left[\text{H}, \ \text{M}, \ \text{0}, \ \text{4}, \ \text{2}, \ \text{1}\right], \\ & \left[\text{H}, \ \text{M}, \ \text{2}, \ \text{4}, \ \text{4}, \ \text{0}\right]\right\} \\ 
\end{aligned} 
\end{equation}

    Finally, we take each element of \(\tilde{\mu}\) in turn and find its most
    similar point in the dataset. This collection of points in the dataset then
    forms our set of initial modes \(\bar{\mu}\) to be passed on to the 
    \(k\)-modes algorithm. We stipulate that no point which is identical to
    another that has been already selected may be used as an initial mode. This
    is done so as to avoid empty clusters further down the line.

    \begin{table}[H]
    \centering
    \singlespacing{%
    \resizebox{.8\textwidth}{!}{%
        \begin{tabular}{lllllllr}
\toprule
{} & Price & Maintenance & Doors & Passengers & Wheels & Eco-Friendly &  Dissimilarity to \$\textbackslashtilde\{\textbackslashmu\}\_1\$ \\
\midrule
5 &     M &           L &     2 &          4 &      4 &            1 &                                 1 \\
7 &     L &           V &     2 &          4 &      4 &            0 &                                 2 \\
3 &     H &           L &     4 &          5 &      4 &            1 &                                 3 \\
4 &     M &           M &     2 &          5 &      4 &            1 &                                 3 \\
0 &     H &           M &     2 &          2 &      4 &            0 &                                 4 \\
1 &     L &           M &     0 &          1 &      2 &            0 &                                 5 \\
2 &     V &           H &     2 &          3 &      8 &            0 &                                 5 \\
6 &     V &           H &     4 &          5 &      4 &            0 &                                 5 \\
8 &     H &           M &     0 &          2 &      2 &            1 &                                 5 \\
9 &     H &           M &     4 &          7 &      4 &            0 &                                 5 \\
\bottomrule
\end{tabular}

    }}
    \caption{The dataset ranked by dissimilarity to the first element of
    \(\tilde{\mu}\).}\label{tab:huang-mode-dissim}
    \end{table}

    Taking the first element of \(\tilde{\mu}\), we calculate the dissimilarity
    between this vector and all of our datapoints.
    Table~\ref{tab:huang-mode-dissim} shows the elements of our dataset ranked
    in ascending order of their dissimilarity to this vector. It follows that we
    should set our first initial mode to be the sixth entry of our dataset. We
    continue this process for the other elements of \(\tilde{\mu}\),
    disregarding any points that have already been selected.
    
    Using our Python implementation for Huang's initialisation method we have 
    that the set of initial modes for this instance of the \(k\)-modes algorithm
    correspond to the sixth, fifth and fourth rows of the dataset. That is, we
    have:

    \begin{equation}
\nonumber
\begin{aligned}
\bar{\mu} = \{  & \left[\text{2}, \ \text{1}, \ \text{L}, \ \text{4}, \ \text{M}, \ \text{4}\right], \\  & \left[\text{2}, \ \text{1}, \ \text{M}, \ \text{5}, \ \text{M}, \ \text{4}\right], \\  & \left[\text{4}, \ \text{1}, \ \text{L}, \ \text{5}, \ \text{H}, \ \text{4}\right]\} \\ 
\end{aligned}
\end{equation}
\end{example}





\subsubsection{Cao's method}\label{subsec:cao}

The second initialisation process that is widely used with \(k\)-modes is known
as Cao's method~\cite{Cao2009}. This method selects representative points
according to their density in the dataset whilst forcing dissimilarity between
them. Definition~\ref{def:density} formalises the concept of density and its
relationship to relative frequency. The method, which is described in
Algorithm~\ref{alg:cao}, is deterministic --- unlike Huang's method which relies
on random sampling.

\begin{definition}\label{def:density}	
    Consider a dataset
    \(\mathcal{X} \subset \mathcal{A} = \{A_1, \ldots, A_m\}\). Then the
    \emph{average~density} of any point \(X_i \in \mathcal{X}\) with respect to
    \(\mathcal{A}\) is defined~\cite{Cao2009} as:
    \begin{equation}\label{eq:density}
        \text{Dens}\left(X^{(i)}\right) = \frac{%
            \sum_{j=1}^m \text{Dens}_{j}\left(X^{(i)}\right)
        }{m}
        \ \ \text{where} \ \
        \text{Dens}_{j}\left(X^{(i)}\right) = \frac{%
            \abs*{%
                \left\{X^{(t)} \in \mathcal{X} : x_j^{(i)} = x_j^{(t)}\right\}
            }
        }{N}
    \end{equation}

    Observe that:
    \[
        \abs*{\left\{X^{(t)} \in \mathcal{X} : x_j^{(i)} = x_j^{(t)}\right\}}%
        = n\left(x_j^{(i)}\right)%
        = \sum_{t=1}^N \left(1 - \delta\left(x_j^{(i)}, x_j^{(t)}\right)\right)
    \]

    And so, an alternative definition for~\eqref{eq:density} can be derived:
    \begin{equation}\label{eq:density-alt}
    \begin{aligned}
        \text{Dens}\left(X^{(i)}\right)
        & = \frac{1}{mN} \sum_{j=1}^m \sum_{t=1}^N \left(%
            1 - \delta\left(x_j^{(i)}, x_j^{(t)}\right)
        \right)\\
        & = \frac{1}{mN} \sum_{j=1}^m \sum_{t=1}^N 1%
            - \frac{1}{mN} \sum_{j=1}^m \sum_{t=1}^N
            \delta\left(x_j^{(i)}, x_j^{(t)}\right)\\
        & = \frac{mN}{mN} - \frac{1}{mN} \sum_{t=1}^N
            d\left(X^{(i)}, X^{(t)}\right)\\
        & = 1 - \frac{1}{mN} D\left(\mathcal{X}, X^{(i)}\right)
    \end{aligned}
    \end{equation}

    % TODO Does this actually make any sense?
    With this alternative definition, it is clear --- since \(m\) and \(N\) are
    fixed positive integers --- that \(\text{Dens}(X^{(i)})\) is maximised when
    \(D(\mathcal{X}, X^{(i)})\) is minimised. Then by Theorem~\ref{thm:mode},
    any data point with maximal average density is, in fact, a mode of
    \(\mathcal{X}\). This observation indicates that there is a similarity
    between this method and Huang's in that they are attempting to achieve the
    same objective if only from opposite ends.
\end{definition}

\begin{algorithm}[H]
\caption{Cao's method}\label{alg:cao}
	\begin{algorithmic}[0]
        \State{\textbf{Input:} a dataset \textbf{X}, with attribute sets \(A_1,
        \ldots, A_m\), and a number of modes to find \(k\)}
        \State{\textbf{Output:} a set of \(k\) initial modes \(\bar{\mu}\)\\}
        \\
        \Comment{Initialisation step}
        \State{\(\bar{\mu} \gets \emptyset\)}
        \For{\(X^{(i)} \in \textbf{X}\)}
            \State{Calculate \(\text{Dens}(X^{(i)})\).}
		\EndFor
        \\
        \\
        \Comment{Select the point with maximal density}
        \State{Select \(X^{(i_1)} \in \textbf{X}\) which satisfies:
        \[
            X^{(i_1)} = \argmax_{1 \leq i \leq N} 
            \left\{\text{Dens}(X^{(i)})\right\}
        \]
        }
        \State{\(\bar{\mu} \gets \bar{\mu} \cup X^{(i_1)}\)\\}
        \\
        \Comment{Second point maximises both density and distance from the first
        mode}
        \State{Select \(X^{(i_2)} \in \textbf{X}\) which satisfies: 
		\[
            X^{(i_2)} = \argmax_{1 \leq i \leq N} \left\{\text{Dens}(X^{(i)})
            \times d(X^{(i)}, X^{(i_1)})\right\}
		\]
        }
        \\
        \State{\(\bar{\mu} \gets \bar{\mu} \cup X^{(i_2)}\)\\}
        \\
        \Comment{Continue to choose points in this fashion until \(k\) are
        chosen}
        \While{\(|\bar{\mu}| < k\)}
            \State{Select \(X^{(i_3)} \in \textbf{X}\) which satisfies:
			\[
                X^{(i_3)} = \argmax_{1 \leq i \leq N} \left\{\min_{\mu^{(l)} \in
                \bar{\mu}} \left\{\text{Dens}(X^{(i)}) \times d(X^{i}, 
                \mu^{(l)})\right\}\right\}
			\]
            }
            \State{\(\bar{\mu} \gets \bar{\mu} \cup X^{(i_3)}\)}
		\EndWhile
	\end{algorithmic}
\end{algorithm}


%\begin{example}\label{ex:cao}
    We will now attempt to find \(3\) initial modes for our vehicle dataset 
    using Cao's method, as we did in Example~\ref{ex:huang}. We begin by 
    calculating the average density of each of our points. We rank these in 
    descending order, and take the point with maximal density as our first 
    initial mode. This ranking is shown in Table~\ref{tab:ranked-density}.

    \begin{table}[H]
        \centering
        \singlespacing{%
        \resizebox{.8\textwidth}{!}{%
            \begin{tabular}{cccccccc}
\toprule
{} & Buying price & Maintenance costs &  No. doors &  No. passengers &  No. wheels &  Eco-friendly &   Density \\
\midrule
1  &            H &                 M &          2 &               2 &           4 &             0 &  0.483333 \\
5  &            M &                 M &          2 &               5 &           4 &             1 &  0.433333 \\
10 &            H &                 M &          4 &               7 &           4 &             0 &  0.433333 \\
4  &            H &                 L &          4 &               5 &           4 &             1 &  0.383333 \\
7  &            V &                 H &          4 &               5 &           4 &             0 &  0.383333 \\
8  &            L &                 V &          2 &               4 &           4 &             0 &  0.383333 \\
6  &            M &                 L &          2 &               4 &           4 &             1 &  0.366667 \\
9  &            H &                 M &          0 &               2 &           2 &             1 &  0.316667 \\
2  &            L &                 M &          0 &               1 &           2 &             0 &  0.300000 \\
3  &            V &                 H &          2 &               3 &           8 &             0 &  0.283333 \\
\bottomrule
\end{tabular}

        }}
        \caption{The dataset ranked by average
            density.}\label{tab:ranked-density}
    \end{table}

    So, from Table~\ref{tab:ranked-density}, we see that the first row should be
    taken as our first initial mode, \(\mu^{(1)}\). This is something we should
    expect since it was seen in Example~\ref{ex:mode} that this entry has
    minimal summed dissimilarity, and from Equation~\ref{eq:alt-def} we know
    that this is equivalent to maximising density.
    
    Now, we wish to find the point which has the maximal product of its density
    and its dissimilarity with our first mode. One way of doing this is to 
    calculate the dissimilarity between each point and the mode, append this
    as a column to our table and multiply these two new columns by each other
    to give \(\text{Dens}(X^{(i)}) \times d(\mu^{(1)}, X^{(i)})\) for each \(i =
    1, \ldots, 10\). The entries are then ranked by this product, and the first
    entry is taken as the second mode. By inspecting 
    Table~\ref{tab:ranked-dens-dissim}, we see that there is a tie. In practical
    implementations we can only assume that ties are broken arbitrarily. So, we
    shall take the fourth row as our second initial mode, \(\mu^{(2)}\).

    \begin{table}[H]
        \singlespacing{%
        \resizebox{\textwidth}{!}{%
            \begin{tabular}{cccccccccc}
\toprule
{} & Buying price & Maintenance costs &  No. doors &  No. passengers &  No. wheels &  Eco-friendly &   Density &  Dissimilarity &  Density-dissimilarity \\
\midrule
4  &            H &                 L &          4 &               5 &           4 &             1 &  0.383333 &              4 &               1.533333 \\
7  &            V &                 H &          4 &               5 &           4 &             0 &  0.383333 &              4 &               1.533333 \\
6  &            M &                 L &          2 &               4 &           4 &             1 &  0.366667 &              4 &               1.466667 \\
5  &            M &                 M &          2 &               5 &           4 &             1 &  0.433333 &              3 &               1.300000 \\
2  &            L &                 M &          0 &               1 &           2 &             0 &  0.300000 &              4 &               1.200000 \\
8  &            L &                 V &          2 &               4 &           4 &             0 &  0.383333 &              3 &               1.150000 \\
3  &            V &                 H &          2 &               3 &           8 &             0 &  0.283333 &              4 &               1.133333 \\
9  &            H &                 M &          0 &               2 &           2 &             1 &  0.316667 &              3 &               0.950000 \\
10 &            H &                 M &          4 &               7 &           4 &             0 &  0.433333 &              2 &               0.866667 \\
1  &            H &                 M &          2 &               2 &           4 &             0 &  0.483333 &              0 &               0.000000 \\
\bottomrule
\end{tabular}

        }}
        \caption{A ranking of the dataset by those who have highest
            density-dissimilarity product with the first
            mode.}\label{tab:ranked-dens-dissim}
    \end{table}

    In order to find the final initial mode, \(\mu^{(3)}\), we actually need to
    find a pair \((X^{(i_3)}, \mu^{(m)})\) as is stated in 
    Algorithm~\ref{alg:cao}. In order to do this, and the process would be the
    same for any further modes, we must consider all of our current initial
    modes, the dissimilarity between each point in our dataset and these modes,
    and the density of each point in the dataset. A convenient way of displaying
    all of this information is to construct a density-dissimilarity matrix which
    we denote by \(\mathbb{D}\) and define as follows:
    \begin{itemize}
        \item \(\mathbb{D}\) has \(|\bar{\mu}|\) rows and \(N\) columns, where
            \(|\bar{\mu}|\) is the number of initial modes already selected.
        \item The entries of \(\mathbb{D}\) are given by:
            \[
                \mathbb{D}_{li} = \text{Dens}(X^{(i)}) \times d(X^{(i)},
                \mu^{(l)}) \ \text{for all} \ l = 1, \ldots, |\bar{\mu}| \
                \text{and} \ i = 1, \ldots, N
            \]
    \end{itemize}

    Now, we go through each column and highlight the smallest value. These
    represent which current mode has minimal density-dissimilarity with the
    \(i^{th}\) datapoint (column). Then, we go through the highlighted entries
    and select the column which has the largest value. This column corresponds
    to the next datapoint to be selected as an initial mode. This process is
    shown in Figure~\ref{fig:cao-matrix}.
    
    \begin{figure}[H]
        \centering
        \singlespacing{%
        \begin{minipage}{\textwidth}
            \centering
            \(
            \begin{pmatrix}
                0 & 1.2 & 1.1\dot{3} & 1.5\dot{3} & 1.3 & 1.4\dot{6} &
                1.5\dot{3} & 1.15 & 0.95 & 0.8\dot{6}
                \\
                1.9\dot{3} & 1.8 & 1.7 & 0 & 1.3 & 1.1 & 1.15 & 1.91\dot{6} &
                1.2\dot{6} & 1.3
            \end{pmatrix}
            \)
        \end{minipage}

        \vspace{10pt}

        \begin{minipage}{\textwidth}
            \centering
            \(
            \begin{pmatrix}
                \underline{0} & \underline{1.2} &
                \underline{1.1\dot{3}} & 1.5\dot{3} & \underline{1.3}
                & 1.4\dot{6} & 1.5\dot{3} & \underline{1.15} &
                \underline{0.95} & \underline{0.8\dot{6}}
                \\
                1.9\dot{3} & 1.8 & 1.7 & \underline{0} &
                \underline{1.3} & \underline{1.1} &
                \underline{1.15} & 1.91\dot{6} & 1.2\dot{6} & 1.3
            \end{pmatrix}
            \)
        \end{minipage}

        \vspace{10pt}

        \begin{minipage}{\textwidth}
            \centering
            \(
            \begin{pmatrix}
                \underline{0} & \underline{1.2} & \underline{1.1\dot{3}} &
                1.5\dot{3} & \textcolor{red}{\underline{1.3}} & 1.4\dot{6} &
                1.5\dot{3} & \underline{1.15} & \underline{0.95} &
                \underline{0.8\dot{6}}
                \\
                1.9\dot{3} & 1.8 & 1.7 & \underline{0} &
                \textcolor{red}{\underline{1.3}} & \underline{1.1} &
                \underline{1.15} & 1.91\dot{6} & 1.2\dot{6} & 1.3
            \end{pmatrix}
            \)
        \end{minipage}
        }
        \caption{The stages of selecting the \(l^{th}\) mode with a
        density-dissimilarity matrix, for \(l > 2\). First, the row with smaller
        value is highlighted in each column (underlined here). Then of those
        highlighted entries, the entry with maximal value is selected (shown in
    red). Ties are broken arbitrarily.}\label{fig:cao-matrix}
    \end{figure}

    Therefore, our set of initial modes, \(\bar{\mu}\), correspond to the first,
    fourth and fifth rows of our dataset. That is:
    
    \begin{equation}
\nonumber
\begin{aligned}
\bar{\mu} = \{  & \left[\text{2}, \ \text{0}, \ \text{M}, \ \text{2}, \ \text{H}, \ \text{4}\right], \\  & \left[\text{4}, \ \text{1}, \ \text{L}, \ \text{5}, \ \text{H}, \ \text{4}\right], \\  & \left[\text{2}, \ \text{1}, \ \text{M}, \ \text{5}, \ \text{M}, \ \text{4}\right]\} \\ 
\end{aligned}
\end{equation}
\end{example}

