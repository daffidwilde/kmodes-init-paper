Now that we have defined what we mean by a matching game, with the algorithm 
described above, we can construct an alternative initialisation process for the 
\(k\)-modes algorithm. \\

Let \textbf{X} be a dataset with attribute set \textbf{A}, and let \(\bar{\mu}\) 
be the set of virtual modes found by Huang's method (i.e.\ the set of centroids 
found in the mode which are to be assigned to points in \textbf{X}). We then 
take this set of virtual modes \(\bar{\mu}\) and construct a capacitated 
matching game to be solved by the capacitated Gale-Shapley algorithm in the 
following way.

\begin{algorithm}[H]
\caption{The proposed initialisation method}
    \begin{algorithmic}[0]
        \State{Find \(\bar{\mu}\) according to Huang's method, up until the 
        final `for' loop.}
        \State{\(R \gets \bar{\mu}\)}
        \State{\(S \gets \emptyset\)}
        \State{\(C \gets \{c_1, \ldots, c_k\}\)}
        \For{\(r \in R\)}
            \State{\(c_r \gets 1\)}
            \State{Find the set of \(k\) vectors, \(S_r\), in \textbf{X} that 
            are the least dissimilar to \(r\).}
            \State{\(S \gets S \cup S_r\)}
        \EndFor
        \For{\(r \in R\)}
        \State{\(g(r) \gets S_r\) (here, \(S_r\) is a tuple in descending order
        of similarity)}
        \EndFor
        \State{Select a method for suitor preference lists (see 
        Section~\ref{sec:prefs}) and construct \(f(s)\) accordingly for each \(s
        \in S\).}
        \State{Solve the capacitated matching game defined by \((S, R, C)\) to
        obtain a matching \(M: R \to S\)}.
        \For{\(r \in R\)}
            \State{\(\mu^{(l)} \gets M(r)\)}.
        \EndFor
    \end{algorithmic}
\end{algorithm}

\begin{remark}
    The method for constructing the preference lists of our suitors can affect
    the outcome and performance of this method. Please refer to
    Sections~\ref{sec:preferences}~\&~\ref{sec:results}.
\end{remark}
