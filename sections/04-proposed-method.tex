Now that we have defined what we mean by a matching game, with the algorithm 
described above, we can construct an alternative initialisation process for the 
$k$-modes algorithm. \\

Let \textbf{X} be a dataset with attribute set \textbf{A}, and let \(\bar{\mu}\) 
be the set of virtual modes found by the Huang method (i.e.\ the set found 
before the selection of points in \textbf{X}). We then take this set of virtual 
modes \(\bar{\mu}\) and construct a capacitated matching game to be solved by 
the capacitated Gale-Shapley algorithm in the following way.

\begin{algorithm}[H]
\caption{The proposed initialisation method}
    \begin{algorithmic}[0]
        \For{Condition}
            \State Something
        \EndFor
    \end{algorithmic}
\end{algorithm}



\begin{itemize}
    \item The set of hospitals \(H\) is \(\bar{\mu}\), and each hospital has 
        capacity \(1\).

    \item The set of residents, \(R\), is made up of the \(k\) least dissimilar 
        points \(X_{l,1}, \ldots, X_{l,k} \in \textbf{X}\) to each \(\mu^{(l)} 
        \in \bar{\mu}\).

	\item Each hospital's preference list is simply their addition to the set of
        residents in descending order of similarity.
	
	\item The preference lists of the residents is more complicated. In this 
        initial implementation, we take their preference list to be the set of 
        hospitals in ascending order of dissimilarity. Though, as will be seen 
        in Section~\ref{section:results}, other ways of generating these lists 
        (such as randomly) can provide different results.
\end{itemize}

Now, by applying the capacitated Gale-Shapley algorithm to this game, we find a 
resident-optimal matching \(M\). Let our set of modes \(\bar{\mu} := 
M^{-1}(H)\). That is, the \(l^{th}\) mode is the resident matched with
\(\mu^{(l)}\) when the algorithm concludes.

