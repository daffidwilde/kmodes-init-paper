\begin{example}\label{ex:cao}
    We will now attempt to find \(3\) initial modes for our vehicle dataset 
    using Cao's method, as we did in Example~\ref{ex:huang}. We begin by 
    calculating the average density of each of our points. We rank these in 
    descending order, and take the point with maximal density as our first 
    initial mode. This ranking is shown in Table~\ref{tab:ranked-density}.

    \begin{table}[H]
        \centering
        \singlespacing{%
        \resizebox{.8\textwidth}{!}{%
            \begin{tabular}{cccccccc}
\toprule
{} & Buying price & Maintenance costs &  No. doors &  No. passengers &  No. wheels &  Eco-friendly &   Density \\
\midrule
1  &            H &                 M &          2 &               2 &           4 &             0 &  0.483333 \\
5  &            M &                 M &          2 &               5 &           4 &             1 &  0.433333 \\
10 &            H &                 M &          4 &               7 &           4 &             0 &  0.433333 \\
4  &            H &                 L &          4 &               5 &           4 &             1 &  0.383333 \\
7  &            V &                 H &          4 &               5 &           4 &             0 &  0.383333 \\
8  &            L &                 V &          2 &               4 &           4 &             0 &  0.383333 \\
6  &            M &                 L &          2 &               4 &           4 &             1 &  0.366667 \\
9  &            H &                 M &          0 &               2 &           2 &             1 &  0.316667 \\
2  &            L &                 M &          0 &               1 &           2 &             0 &  0.300000 \\
3  &            V &                 H &          2 &               3 &           8 &             0 &  0.283333 \\
\bottomrule
\end{tabular}

        }}
        \caption{The dataset ranked by average
            density.}\label{tab:ranked-density}
    \end{table}

    So, from Table~\ref{tab:ranked-density}, we see that the first row should be
    taken as our first initial mode, \(\mu^{(1)}\). This is something we should
    expect since it was seen in Example~\ref{ex:mode} that this entry has
    minimal summed dissimilarity, and from Equation~\ref{eq:alt-def} we know
    that this is equivalent to maximising density.
    
    Now, we wish to find the point which has the maximal product of its density
    and its dissimilarity with our first mode. One way of doing this is to 
    calculate the dissimilarity between each point and the mode, append this
    as a column to our table and multiply these two new columns by each other
    to give \(\text{Dens}(X^{(i)}) \times d(\mu^{(1)}, X^{(i)})\) for each \(i =
    1, \ldots, 10\). The entries are then ranked by this product, and the first
    entry is taken as the second mode. By inspecting 
    Table~\ref{tab:ranked-dens-dissim}, we see that there is a tie. In practical
    implementations we can only assume that ties are broken arbitrarily. So, we
    shall take the fourth row as our second initial mode, \(\mu^{(2)}\).

    \begin{table}[H]
        \singlespacing{%
        \resizebox{\textwidth}{!}{%
            \begin{tabular}{cccccccccc}
\toprule
{} & Buying price & Maintenance costs &  No. doors &  No. passengers &  No. wheels &  Eco-friendly &   Density &  Dissimilarity &  Density-dissimilarity \\
\midrule
4  &            H &                 L &          4 &               5 &           4 &             1 &  0.383333 &              4 &               1.533333 \\
7  &            V &                 H &          4 &               5 &           4 &             0 &  0.383333 &              4 &               1.533333 \\
6  &            M &                 L &          2 &               4 &           4 &             1 &  0.366667 &              4 &               1.466667 \\
5  &            M &                 M &          2 &               5 &           4 &             1 &  0.433333 &              3 &               1.300000 \\
2  &            L &                 M &          0 &               1 &           2 &             0 &  0.300000 &              4 &               1.200000 \\
8  &            L &                 V &          2 &               4 &           4 &             0 &  0.383333 &              3 &               1.150000 \\
3  &            V &                 H &          2 &               3 &           8 &             0 &  0.283333 &              4 &               1.133333 \\
9  &            H &                 M &          0 &               2 &           2 &             1 &  0.316667 &              3 &               0.950000 \\
10 &            H &                 M &          4 &               7 &           4 &             0 &  0.433333 &              2 &               0.866667 \\
1  &            H &                 M &          2 &               2 &           4 &             0 &  0.483333 &              0 &               0.000000 \\
\bottomrule
\end{tabular}

        }}
        \caption{A ranking of the dataset by those who have highest
            density-dissimilarity product with the first
            mode.}\label{tab:ranked-dens-dissim}
    \end{table}

    In order to find the final initial mode, \(\mu^{(3)}\), we actually need to
    find a pair \((X^{(i_3)}, \mu^{(m)})\) as is stated in 
    Algorithm~\ref{alg:cao}. In order to do this, and the process would be the
    same for any further modes, we must consider all of our current initial
    modes, the dissimilarity between each point in our dataset and these modes,
    and the density of each point in the dataset. A convenient way of displaying
    all of this information is to construct a density-dissimilarity matrix which
    we denote by \(\mathbb{D}\) and define as follows:
    \begin{itemize}
        \item \(\mathbb{D}\) has \(|\bar{\mu}|\) rows and \(N\) columns, where
            \(|\bar{\mu}|\) is the number of initial modes already selected.
        \item The entries of \(\mathbb{D}\) are given by:
            \[
                \mathbb{D}_{li} = \text{Dens}(X^{(i)}) \times d(X^{(i)},
                \mu^{(l)}) \ \text{for all} \ l = 1, \ldots, |\bar{\mu}| \
                \text{and} \ i = 1, \ldots, N
            \]
    \end{itemize}

    Now, we go through each column and highlight the smallest value. These
    represent which current mode has minimal density-dissimilarity with the
    \(i^{th}\) datapoint (column). Then, we go through the highlighted entries
    and select the column which has the largest value. This column corresponds
    to the next datapoint to be selected as an initial mode. This process is
    shown in Figure~\ref{fig:cao-matrix}.
    
    \begin{figure}[H]
        \centering
        \singlespacing{%
        \begin{minipage}{\textwidth}
            \centering
            \(
            \begin{pmatrix}
                0 & 1.2 & 1.1\dot{3} & 1.5\dot{3} & 1.3 & 1.4\dot{6} &
                1.5\dot{3} & 1.15 & 0.95 & 0.8\dot{6}
                \\
                1.9\dot{3} & 1.8 & 1.7 & 0 & 1.3 & 1.1 & 1.15 & 1.91\dot{6} &
                1.2\dot{6} & 1.3
            \end{pmatrix}
            \)
        \end{minipage}

        \vspace{10pt}

        \begin{minipage}{\textwidth}
            \centering
            \(
            \begin{pmatrix}
                \underline{0} & \underline{1.2} &
                \underline{1.1\dot{3}} & 1.5\dot{3} & \underline{1.3}
                & 1.4\dot{6} & 1.5\dot{3} & \underline{1.15} &
                \underline{0.95} & \underline{0.8\dot{6}}
                \\
                1.9\dot{3} & 1.8 & 1.7 & \underline{0} &
                \underline{1.3} & \underline{1.1} &
                \underline{1.15} & 1.91\dot{6} & 1.2\dot{6} & 1.3
            \end{pmatrix}
            \)
        \end{minipage}

        \vspace{10pt}

        \begin{minipage}{\textwidth}
            \centering
            \(
            \begin{pmatrix}
                \underline{0} & \underline{1.2} & \underline{1.1\dot{3}} &
                1.5\dot{3} & \textcolor{red}{\underline{1.3}} & 1.4\dot{6} &
                1.5\dot{3} & \underline{1.15} & \underline{0.95} &
                \underline{0.8\dot{6}}
                \\
                1.9\dot{3} & 1.8 & 1.7 & \underline{0} &
                \textcolor{red}{\underline{1.3}} & \underline{1.1} &
                \underline{1.15} & 1.91\dot{6} & 1.2\dot{6} & 1.3
            \end{pmatrix}
            \)
        \end{minipage}
        }
        \caption{The stages of selecting the \(l^{th}\) mode with a
        density-dissimilarity matrix, for \(l > 2\). First, the row with smaller
        value is highlighted in each column (underlined here). Then of those
        highlighted entries, the entry with maximal value is selected (shown in
    red). Ties are broken arbitrarily.}\label{fig:cao-matrix}
    \end{figure}

    Therefore, our set of initial modes, \(\bar{\mu}\), correspond to the first,
    fourth and fifth rows of our dataset. That is:
    
    \begin{equation}
\nonumber
\begin{aligned}
\bar{\mu} = \{  & \left[\text{2}, \ \text{0}, \ \text{M}, \ \text{2}, \ \text{H}, \ \text{4}\right], \\  & \left[\text{4}, \ \text{1}, \ \text{L}, \ \text{5}, \ \text{H}, \ \text{4}\right], \\  & \left[\text{2}, \ \text{1}, \ \text{M}, \ \text{5}, \ \text{M}, \ \text{4}\right]\} \\ 
\end{aligned}
\end{equation}
\end{example}
