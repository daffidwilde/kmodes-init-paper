\begin{example}\label{ex:dissim}
    Throughout this work, we will make use of a number of small examples to aid
    our understanding of various concepts. These examples will utilise a small,
    artificial dataset describing some qualities about vehicles.
    
    The dataset is made up of \(N = 10\) instances, each of which describe a
    vehicle. These instances are defined by \(m = 6\) attributes, the first two
    of which are ordinal variables taken from the set \(\{\text{L, M, H, V}\}\)
    standing for `low', `medium', `high', and `very high' respectively. The
    next three are integer variables and so can be considered as categorical,
    and the final attribute is a binary variable indicating whether the vehicle
    is eco-friendly \((1)\) or not \((0)\). The full dataset is given in
    Table~\ref{tab:dataset}. Please note that there is an additional, unheaded
    column on the left hand side showing the index starting at \(1\) and going
    up to \(5\).
    
    \begin{table}[H]
        \centering
        \singlespacing{%
        \resizebox{.8\textwidth}{!}{%
            \centering
            \begin{tabular}{ccccccc}
\toprule
{} & Buying price & Maintenance costs & No. doors & No. passengers & No. wheels & Eco-friendly \\
\midrule
1  &            H &                 M &         2 &              2 &          4 &            0 \\
2  &            L &                 M &         0 &              1 &          2 &            0 \\
3  &            V &                 H &         2 &              3 &          8 &            0 \\
4  &            H &                 L &         4 &              5 &          4 &            1 \\
5  &            M &                 M &         2 &              5 &          4 &            1 \\
6  &            M &                 L &         2 &              4 &          4 &            1 \\
7  &            V &                 H &         4 &              5 &          4 &            0 \\
8  &            L &                 V &         2 &              4 &          4 &            0 \\
9  &            H &                 M &         0 &              2 &          2 &            1 \\
10 &            H &                 M &         4 &              7 &          4 &            0 \\
\bottomrule
\end{tabular}

        }}
        \caption{The vehicle dataset.}\label{tab:dataset}
    \end{table}

    Let us consider our first two datapoints. With the notation laid out in
    Section~\ref{subsec:notation}, we can express these points as vectors in the
    following way:

    \begin{equation}
        \nonumber
        \begin{aligned}
            X^{(1)} & = & \left[ x_1^{(1)} = \text{H}, \ x_2^{(1)} = \text{M}, \
            x_3^{(1)} = 2, \ x_4^{(1)} = 2, \ x_5^{(1)} = 4, \ x_6^{(1)} = 0 
            \right]
            \\
            X^{(2)} & = & \left[ x_1^{(2)} = \text{L}, \ x_2^{(2)} = \text{M}, \
            x_3^{(2)} = 0, \ x_4^{(2)} = 1, \ x_5^{(2)} = 2, \ x_6^{(2)} = 0
            \right]
        \end{aligned}
    \end{equation}

    Then, by Definition~\ref{def:dissim}, their pairwise dissimilarity is:
    \begin{equation}
        \nonumber
        \begin{aligned}
            \centering
            d(X^{(1)}, X^{(2)}) & = & \delta(\text{H}, \text{L}) & + & 
            \delta(\text{M}, \text{M}) & + & \delta(3, 0) & + & \delta(2, 1) &
            + & \delta(4, 2) & + & \delta(0, 0) & {} & {}
            \\
            {} & = & 1 \ & + & 0 \ & + & 1 \ & + & 1 \ & + & 1 \ & + & 0 \ & = &
            4
        \end{aligned}
    \end{equation}
\end{example}
